\documentclass[11pt,a4paper,dvipsnames,usenames]{beamer}
%\documentclass[11pt,a4paper,dvipsnames,usenames,handout]{beamer}
\usepackage{lmodern}
\usepackage{url}
\usepackage{xcolor}
\usepackage{amsmath,varwidth}
\usepackage[utf8]{inputenc}
\usepackage[T1]{fontenc}
\usepackage{tikz,tikz-uml,xparse}
\usepackage{listings,algpseudocode}
\usepackage[normalem]{ulem}
\usepackage{transparent}

% -------------------------- %
% Standard LaTeX definitions %
% -------------------------- %

\DeclareGraphicsExtensions{.pdf}
\renewcommand\textbullet{\ensuremath{\bullet}}

% ----------------------------------- %
% Various tikz styles and definitions %
% ----------------------------------- %

\usetikzlibrary{calc,intersections,positioning,matrix,chains,scopes}
\usetikzlibrary{decorations.pathmorphing,decorations.pathreplacing,shapes}
\newcommand{\tikzmark}[1]{\tikz[overlay,remember picture] \node (#1) {};}

\tikzumlset{fill class=RoyalBlue!20}

\tikzstyle{startnode} = [circle, draw, fill=Bittersweet!20, minimum size=10pt, inner sep = 0]
\tikzstyle{stopnode} = [startnode]
\tikzstyle{process} = [rectangle, minimum width=2cm, minimum height=.7cm, text centered, draw, fill=OliveGreen!20, align=center, font=\scriptsize]
\tikzstyle{dangerprocess} = [process, fill=BrickRed!20, draw=BrickRed, thick]
\tikzstyle{flowarrow} = [->, >=stealth, thick]

% ---------------------------------- %
% My default listings C++ code style %
% ---------------------------------- %

\lstset{%
  language=C++, basicstyle=\scriptsize\ttfamily, 
  keywordstyle=\color{OliveGreen}, identifierstyle=\color{RoyalBlue}, 
  commentstyle=\color{Brown}, stringstyle=\color{Bittersweet}, showstringspaces=false,  
  breaklines=true, prebreak=\mbox{{\color{Bittersweet}\tiny\ $\hookleftarrow$}}, 
  postbreak=\mbox{{\color{Bittersweet}\tiny$\to$\ }}, tabsize=5,
  morekeywords={%
    MyClass,unique_ptr,shared_ptr,weak_ptr,auto_ptr,
    make_unique, make_shared, default_delete,
    dynamic_pointer_cast, const_pointer_cast,
    scoped_ptr,QSharedPointer,QScopedPointer,QWeakPointer,
    ftring, SmartPtr, Shape, Circle, ShapeFactory, CircleFactory,
    Derived, Base
  }
}

\def\inline{\lstinline[basicstyle=\ttfamily\normalsize]}

% ---------------------------------------------------------------------------------------------------- %
% Various command for inline code instead of lstinline so that I don't have to update the keyword list %
% ---------------------------------------------------------------------------------------------------- %

\newcommand{\object}[1]{{\ttfamily \color{OliveGreen}#1}}
\newcommand{\function}[1]{{\ttfamily \color{RoyalBlue}#1}}
\newcommand{\std}[1]{{\ttfamily {\color{RoyalBlue} std::}{\color{OliveGreen}#1}}}
\newcommand{\boost}[1]{{\ttfamily {\color{RoyalBlue} boost::}{\color{OliveGreen}#1}}}

\newcommand{\uniqueptr}{\std{unique\_ptr}}
\newcommand{\sharedptr}{\std{shared\_ptr}}
\newcommand{\weakptr}{\std{weak\_ptr}}
\newcommand{\autoptr}{\std{auto\_ptr}}

% The base16 colour scheme?
\definecolor{sbase03}{HTML}{002B36}
\definecolor{sbase02}{HTML}{073642}
\definecolor{sbase01}{HTML}{586E75}
\definecolor{sbase00}{HTML}{657B83}
\definecolor{sbase0}{HTML}{839496}
\definecolor{sbase1}{HTML}{93A1A1}
\definecolor{sbase2}{HTML}{EEE8D5}
\definecolor{sbase3}{HTML}{FDF6E3}
\definecolor{syellow}{HTML}{B58900}
\definecolor{sorange}{HTML}{CB4B16}
\definecolor{sred}{HTML}{DC322F}
\definecolor{smagenta}{HTML}{D33682}
\definecolor{sviolet}{HTML}{6C71C4}
\definecolor{sblue}{HTML}{268BD2}
\definecolor{scyan}{HTML}{2AA198}
\definecolor{sgreen}{HTML}{859900}

\definecolor{Tropiteal}{RGB}{0,168,198}
\definecolor{TealDrop}{RGB}{64,192,203}
\definecolor{WhiteTrash}{RGB}{249,242,231}
\definecolor{AtomicBikini}{RGB}{174,226,57}
\definecolor{FeebleWeek}{RGB}{143,190,0}
\definecolor{ICantExpress}{RGB}{28,20,13}
\definecolor{Marty}{RGB}{250,42,0}

\colorlet{ColourBase}{Tropiteal}
\colorlet{ColourHl1}{Marty}
\colorlet{ColourHl2}{FeebleWeek}
\colorlet{ColourHl3}{TealDrop}
\colorlet{ColourDark}{ICantExpress}
\colorlet{ColourDark2}{Tropiteal}


\usetheme{metropolis}

\setbeamercolor{alerted text}{%
  fg=bazelGreen
}

\setbeamercolor{example text}{%
  fg=mLightBrown
}

\setbeamercolor{frametitle}{%
  use=normal text,
  fg=normal text.bg,
  bg=bazelGreen
}

\lstset{%
  basicstyle=\ttfamily\lst@ifdisplaystyle\fontsize{9pt}{9pt}\selectfont\fi,
  keywordstyle=\color{sgreen}, identifierstyle=\color{sblue}, 
  commentstyle=\color{sbase1}, stringstyle=\color{sorange},
  numberstyle=\color{sviolet}, showstringspaces=false,  
  breaklines=true,
  tabsize=5,
}

\lstalias[]{gnumake}[gnu]{make}

\newcommand{\dimtext}[2]%
{
  { \transparent{0.7}
  \begin{tikzpicture}[overlay, remember picture]
    \fill[white] ( #1 -| current page.north west) -- ++(0,.8em) -- ++(\paperwidth,0) -- (#2 -| current page.north east)
   -- ++(0,-.5em) -- ++(-\paperwidth,0) -- cycle;
  \end{tikzpicture}
  }
}

\newcommand{\specialcell}[2][c]{%
  \begin{tabular}
    [#1]{@{}c@{}}#2
  \end{tabular}
}

\newcommand{\tikzmark}[1]{\tikz[overlay,remember picture] \coordinate (#1);}

\newcommand{\qmat}{\ensuremath{Q_{\mathrm{uark}}}}
\newcommand{\mathfont}{\fontsize{10pt}{12pt}}
\newcommand{\minus}{\scalebox{0.75}[1.0]{$-$}}
\newcommand{\sectionframe}{%
  \addtocounter{framenumber}{-1}%
  \frame[plain]{\begin{center}\LARGE \color{beameralert} \insertsection\end{center}}}

\newcommand{\breakline}{%
  \begin{center} \begin{tikzpicture}
      \draw[beamerprimary] ({-0.5 * \textwidth},0) -- ({0.5 * \textwidth},0);
      \node[inner sep=0, minimum size=7pt, fill=white,circle] {};
      \node[inner sep=0, minimum size=4pt, draw=beamerprimary,circle] {};
      \node[inner sep=0, minimum size=1pt, fill=beamerprimary,circle] {};
  \end{tikzpicture} \end{center}
}


%\setbeameroption{show notes}

\title[TMP]{Template Meta Programming}
\author{Jonas R. Glesaaen \\ \texttt{glesaaen@th.physik.uni-frankfurt.de}}
\date{May 27th 2015}

\begin{document}

\nocite{*}

\begin{frame}
  \titlepage
\end{frame}

\begin{frame}
  \frametitle{Literature}
  \bibliographystyle{plain}
  {\footnotesize
    \bibliography{ref}}
\end{frame}

\begin{frame}[fragile]
  \frametitle{What is Template Meta Programming?}

  \vfill

  Programming using the template interface of C++ so that certain common computations can be carried out at compile time.

  \vfill

  The "language" is functional in nature, no mutable data.

  \vfill

  Advantages:

  \begin{itemize}
    \item Reduce code duplication
    \item Increase readability
    \item Move error checks to compile time
    \item More sophisticated type checking and lookup
  \end{itemize}

  \vfill

  \note[item]
  {
    It is the next step in reducing code duplication over templates. While templates by themselves do a good job,
    using meta programming we can add additional checks and requirements on our types before choosing specialisations.
  }

  \note[item]
  {
    Increase readability is a bit of an odd one as TMP is quite hard to read in my opinion. However, meta programming
    can be used to for example inline horrible low level algorithms into your code for specialised cases, and do for
    example loop unravelling and such. Leaving the code nice and readable while still being efficient at runtime.
  }

  \note[item]
  {
    Always great to move error checks to happen as early as absolutely possible. Specially if a try-block is in a rarely
    executed part of your program. Also see our discussions on physics units in C++ to prevent other types of mistakes.
  }

  % Idea: talk a bit about its history here if time permits

\end{frame}

\begin{frame}[fragile]
  \frametitle{Looks familiar?}
  \framesubtitle{Type traits}

  \begin{lstlisting}
template <class Itt>
void iterator_swap (Itt first, Itt second)
{
  typedef typename std::iterator_traits<Itt>::value_type iterator_deref_type;

  iterator_deref_type temp_value = *first;

  *first = *second;
  *second = temp_value;
}
  \end{lstlisting}

  \note
  {
    First, ignore the fact that this is horribly optimized, and the fact that the problem here
    is easily solved by \object{auto}, also that it could be circumvented by using \inline{std::swap}.

    \vspace{1em}

    Also, \inline{auto} doesn't necessarily give what you want when you have a proxy object with a conversion
    as \inline{auto} will create an object of the same time, circumventing the conversion.

    \vspace{1em}

    We ask the compiler to find out what type dereferencing the iterator gets us using the handy
    \inline{iterator_traits} function. The iterator itself do not necessarily contain this information
    so \inline{iterator_traits} must be specialised for say raw pointers. More on that in a bit.
  }

\end{frame}

\begin{frame}[fragile]
  \frametitle{Looks familiar?}
  \framesubtitle{\inline{enable\_if} (C++14)}

  \begin{lstlisting}
template <
  class Itt,
  typename = std::enable_if_t<
    !std::is_same<
      typename std::iterator_traits<Itt>::value_type,
      void
     >::value
  > 
>
void iterator_function (Itt first, Itt second)
{
  // ...
}
  \end{lstlisting}

  \note
  {
    Ignoring the SFINAE thingy going on, we basically ask the compiler to calculate the expression
    \inline{!(Itt::value_type == void)} at compile time.
  }

\end{frame}

\begin{frame}[fragile]
  \frametitle{Recap: Template Specialisation}

  \vfill

  Heavily used in TMP to signal return paths and branch points for control structures.

  \vfill

  \begin{exampleblock}{\inline{iterator\_traits}}
  \begin{lstlisting}
template <class Itt>
struct iterator_traits
{
  typedef typename Itt::value_type value_type;
};

template <class Type*>
struct iterator_traits
{
  typedef Type value_type;
};
  \end{lstlisting}

  \note
  {
    \inline{iterator\_traits}' existance is motivated by the fact that you can't call \inline{::value\_type} on a
    pointer. On top of that it is also useful in its own right, if only to create a unified interface for meta functions.

    \vspace{1em}

    The Fundamental Theorem of Software Engineering (FTSE). We can solve any problem by introducing an extra level of indirection.
    (Butler Lampson)
  }
\end{exampleblock}

  \vfill

\end{frame}

\begin{frame}[fragile]
  \frametitle{When do you need \object{typename}?}

  \vfill

  \object{typename} is used to tell the compiler that what is coming up is a type. Used when you have 
  a \textbf{dependent} name.

  \vfill

  \begin{exampleblock}{\object{typename} keyword}
    \begin{lstlisting}
    template <class Type>
    typename traits_func<Type>::value_type //...
    \end{lstlisting}
  \end{exampleblock}

  \vfill

  Exactly what \inline{traits_func<Type>::value_type} is cannot be known at point of definition because of possible
  template specialisation. \object{typename} fixes that issue.

  \vfill

  \inline{::value_type} is said to be a dependent type.

  \note
  {
    Because of this it is generally a good idea to use the \object{class} keyword when giving a template 
    a name so that it doesn't get mixed with typename's needed for dependent names.
  }

  \vfill

\end{frame}

\begin{frame}[fragile]
  \frametitle{When do you need \object{template}?}

  \vfill

  If the template class itself is a template, or has a template function, we need to tell the compiler.

  \vfill

  \begin{onlyenv}<1|handout:1>
  \begin{exampleblock}{\object{template} keyword}
    \begin{lstlisting}
    template <class Type, unsigned N>
    void foo(int x)
    {
      Type::function<N>(x);
    };
    \end{lstlisting}
  \end{exampleblock}

  \vfill

  which is interpreted as

  \begin{lstlisting}
    (Type::function < N) > x;
  \end{lstlisting}
  \end{onlyenv}

  \begin{onlyenv}<2|handout:|handout:2>
  \begin{exampleblock}{\object{template} keyword}
    \begin{lstlisting}[escapeinside={(*}{*)}]
    template <class Type, unsigned N>
    void foo(int x)
    {
      Type::(*\tikzmark{template1}*)template(*\tikzmark{template2}*) function<N>(x);
    };
    \end{lstlisting}
  \end{exampleblock}

  \begin{tikzpicture}[overlay, remember picture]
    \draw[decoration={snake,amplitude=.5pt,segment length=3pt,post length=0pt},decorate, RoyalPurple] 
      ($ (template1) + (0,-.1cm)$) -- ( $ (template2) + (0,-.1cm) $);
  \end{tikzpicture}

  \vfill

  \object{template} is required when a \textbf{dependent} name access a template via \inline{.}, \inline{->} or \inline{::}.
  \end{onlyenv}

  \vfill

\end{frame}

\begin{frame}[fragile]
  \frametitle{The Canonical Example}
 
  \begin{lstlisting}[escapeinside={(*}{*)}]
    template<unsigned n>
    struct Factorial
    {
      enum { value = n * Factorial<n-1>::value };
    };

    template<>
    struct Factorial<0>
    {
      enum { value = 1 };
    };

    int main(int, char**)
    {
      std::cout << (*\tikzmark{canonical1}*)Factorial<10>::value(*\tikzmark{canonical2}*) << std::endl;
    }
  \end{lstlisting}

  \uncover<2->{%
  \begin{tikzpicture}[overlay, remember picture]
    \draw[decoration={snake,amplitude=.5pt,segment length=3pt,post length=0pt},decorate, RoyalPurple] 
      ($ (canonical1) + (0,-.2cm)$) -- ( $ (canonical2) + (0,-.2cm) $)
      node [midway,below=.1cm,black,font=\footnotesize] {Runtime constant};
  \end{tikzpicture}
  }

  \note<2>
  {
    Make use of the trick that for the compiler to be able to give you an object of type \inline{Factorial<10>} it
    first need to initialise an object of type \inline{Factorial<9>}, and so on. It continues until it hits the
    template specialisation \inline{Factorial<0>} which we have predefined and requires no more initialisations.
  }

\end{frame}

\begin{frame}[fragile]
  \frametitle{Vocabulary}

  \vfill

  \textbf{Metadata}

  \hfill
  \begin{minipage}{\dimexpr\textwidth-1cm} \small
    A constant "value" accessible by calling \inline{::value}
  \end{minipage}

  \vfill

  \textbf{Metafunction}

  \vspace{10pt}

  \hfill
  \begin{minipage}{\dimexpr\textwidth-1cm} \small
    A function which takes its arguments as template arguments, and the result is stored in \inline{::type}
    \begin{lstlisting}
    some_metafunction<Arg1, Arg2>::type
    \end{lstlisting}
  \end{minipage}

  \vfill

  \textbf{Metafunction class}

  \vspace{10pt}

  \hfill
  \begin{minipage}{\dimexpr\textwidth-1cm} \small
    A function object that itself can be treated as a type. Function call accessed by a nested metafunction named
    \inline{apply}
    \begin{lstlisting}
    struct some_metafunction
    {
      template <class Arg1, class Arg2>
      struct apply
      {
        // ...
      };
    };
    \end{lstlisting}
  \end{minipage}

  \note
  {
    Metadata: Also known as an integral constant wrapper.

    \vspace{1em}

    The metafunction class is the metaprogramming version of a functor, what you would pass to 
    say STL library iterators such as \inline{std::for_each} and \inline{std::copy_if}. In metaprogramming
    this is even more important than normal as we rely much more heavily on passing function objects around
    when we have no mutable data.
  }

  \vfill

\end{frame}

\begin{frame}[fragile]
  \frametitle{Example: Multiplication}

  \begin{onlyenv}<1|handout:1>
  \begin{lstlisting}
template <int N>
struct integer
{
  constexpr static int value = N;
  typedef integer type;
};

template <class Arg1, class Arg2>
struct multiply
{
  typedef integer< Arg1::value * Arg2::value > type;
};

int main(int, argc**)
{
  typedef integer<5> five;
  typedef integer<-9> m_nine;

  std::cout << multiply<five,m_nine>::type::value 
    << std::endl;
}
  \end{lstlisting}
  \end{onlyenv}

  \begin{onlyenv}<2|handout:2>
    \begin{lstlisting}[escapeinside={(*}{*)}]
template <int N>(*\tikzmark{mult1b}*)
struct integer
{
  constexpr static int value = N;
  typedef integer type;
};(*\tikzmark{mult1e}*)

template <class Arg1, class Arg2>(*\tikzmark{multff1}*)
struct multiply
  : integer< Arg1::value * Arg2::value >
{};(*\tikzmark{multff2}*)

int main(int, argc**)(*\tikzmark{mult2b}*)
{
  typedef integer<5> five;
  typedef integer<-9> m_nine;

  std::cout << multiply<five,m_nine>::type::value 
    << std::endl;(*\tikzmark{mult2e}*)

  std::cout << multiply<five,m_nine>::value 
    << std::endl;
}(*\tikzmark{mult3}*)
  \end{lstlisting}

  \dimtext{mult1b}{mult1e}
  \dimtext{mult2b}{mult2e}
  \dimtext{mult3}{mult3}

  \begin{tikzpicture}[overlay, remember picture]
    \coordinate (multitop) at ($(multff1) + (2cm,.6em)$);
    \draw[decorate,decoration=brace] (multitop) -- (multitop |- multff2)
      node[midway,right=.5cm,font=\footnotesize,align=center] {Metafunction \\ forwarding};
  \end{tikzpicture}

  \note<2|handout:2>
  {
    Function definition through inheritance is called metafunction forwarding, and it is one of the resons why we
    defined the type of the integer wrapper, so that we can more easily keep the language consistent.

    \vspace{1em}

    See what we can access its value both through \inline{multiply<x,y>::type::value} and \inline{multiply<x,y>::value}. The first
    interface is expected to be there, while the second is a shortcut sometimes implemented into numerical metafunctions.
  }

  \end{onlyenv}

\end{frame}

\begin{frame}[fragile]
  \frametitle{Higher Order Metafunctions}

  \vfill

  As TMP inherently is a functional programming language, it is best at doing those kind of computations, computations with
  functions.

  \vfill
  
  Let us implement the \texttt{nest} function so that:

  \vfill

  \hspace{1cm} \texttt{nest(f,x,5) = f(f(f(f(f(x)))))}

  \vfill

  Assume that the \inline{integer} and \inline{multiply} still are defined as previous.

  \vfill

\end{frame}

\begin{frame}[fragile]
  \frametitle{Higher Order Metafunctions}

  \vfill

  \begin{lstlisting}[escapeinside={(*}{*)}]
template <class F, class X, unsigned N>(*\tikzmark{nestF1b}*)
struct nest
  : nest<F, (*\tikzmark{nest5}*)typename(*\tikzmark{nest6}*) F::(*\tikzmark{nest7}*)template(*\tikzmark{nest8}*) apply<X>::type, N-1>
{};(*\tikzmark{nestF1e}*)

template <class F, class X> (*\tikzmark{nest1}*)
struct nest <F,X,0>
  : X
{}; (*\tikzmark{nest2}*)

struct squared_f (*\tikzmark{nest3}*)
{
  template <class Arg>
  struct apply
    : multiply<Arg,Arg>
  {};
}; (*\tikzmark{nest4}*)

int main(int, char**)(*\tikzmark{nestF2b}*)
{
  typedef integer<5> five;
  nest<squared_f,five,3>::type::value; // ((5^2)^2)^2 
}(*\tikzmark{nestF2e}*)
  \end{lstlisting}

  \only<2|handout:0>{%
    \dimtext{nestF1b}{nestF1e}
    \dimtext{nestF2b}{nestF2e}
  }

  \only<3|handout:0>{%
    \dimtext{nest1}{nestF2e}
  }

  \uncover<2>{%
  \begin{tikzpicture}[overlay, remember picture]
    \coordinate (n1s) at ($(nest1) + (1em,.6em)$);
    \draw[decorate,decoration=brace] (n1s) -- (n1s |- nest2)
      node[midway,right=1cm,font=\footnotesize,align=center] {Template \\ specialisation};
    \coordinate (n2s) at ($(nest3) + (1em,.6em)$);
    \coordinate (n3s) at (n2s -| n1s);
    \draw[decorate,decoration=brace] (n3s) -- (n3s |- nest4)
      node[midway,right=1cm,font=\footnotesize,align=center] {Metafunction \\ class};
  \end{tikzpicture}
  }

  \uncover<3>{%
  \begin{tikzpicture}[overlay, remember picture]
    \draw[decoration={snake,amplitude=.5pt,segment length=3pt,post length=0pt},decorate, RoyalPurple] 
      ($ (nest5) + (0,-.1cm)$) -- ( $ (nest6) + (0,-.1cm) $);
    \draw[decoration={snake,amplitude=.5pt,segment length=3pt,post length=0pt},decorate, RoyalPurple] 
      ($ (nest7) + (0,-.1cm)$) -- ( $ (nest8) + (0,-.1cm) $);
  \end{tikzpicture}
  }

  \note<3>
  {
    First notice that we need to pass a metafunction class to \inline{nest} as a metafunction without arguments isn't a type
    in itself. We cannot write \inline{nest<multiply,Var,N>} because \inline{multiply} isn't defined without it's template
    arguments. We will see in a bit that we can do exactly that (or something similar) using MPL placeholders, but more on that
    later.
  }

\end{frame}

\begin{frame}[fragile]
  \frametitle{The MPL boost library}

  Collection of useful types and definitions to simplify TMP

  \begin{itemize}
    \item<1-> Metadata wrappers:
      \begin{itemize}
        \item \inline{bool_<b>}, \inline{int_<N>}, \inline{long_<N>}, \dots
      \end{itemize}
    \item<1-> Arithmetic functions and logic operators:
      \begin{itemize}
        \item \inline{plus<Arg1,Arg2>}, \inline{times<Arg1,Arg2>}, \dots
        \item \inline{less<Arg1,Arg2>}, \inline{equal_to<Arg1,Arg2>}, \dots
        \item \inline{and_<Arg>}, \inline{or_<Arg>}, \inline{nor_<Arg>}
      \end{itemize}
    \item<2-> Lambda functions and placeholders
    \item<3-> Type selection
      \begin{itemize}
        \item \inline{if_<Pred,Func1,Func2>}, \inline{eval_if<Pred,Func1,Func2>}
      \end{itemize}
    \item<4-> Containers and iterators
      \begin{itemize}
        \item \inline{vector<Arg1,Arg2,...,ArgN>}, \inline{set<Arg1,Arg2,...,ArgN>}, \dots
        \item \inline{next<It>}, \inline{prior<It>}, \inline{advance<It,N>}, \dots
      \end{itemize}
    \item<4-> STL like algorithm library
      \begin{itemize}
        \item \inline{transform<Seq,Fun>}, \inline{copy_if<Seq,Pred>}, \dots
      \end{itemize}
  \end{itemize}

  \note<4>
  {
    Functions such as \inline{if_<Pred,Func1,Func2>} takes as an argument a metadata object and returns the first
    type \inline{Func1} if \inline{Pred::type::value} is \inline{true} and the type \inline{Func2} if it is
    \inline{false}. There are also integer version of most such functions, denoted by an \inline{_c}, so that e.g.
    \inline{if_c<bool,Func1,Func2>} can be passed a bool instead of a metadata for convenience.

    \vspace{1em}

    Remember that TMP has no mutable objects, so e.g. \inline{transform} will return the transformed sequence rather
    than copying it to another sequece object (or itself).
  }
\end{frame}

\begin{frame}[fragile]
  \frametitle{Lambda functions and placeholders}

  \vfill

  \begin{onlyenv}<1|handout:1>

  First!

  \vspace{1em}

  We will assume that we have the following header on all our code to reduce the examples:
  \begin{lstlisting}
  namespace mpl = boost::mpl;
  using namespace mpl::placeholders;
  \end{lstlisting}

  If not, we would have to write the following everywhere we wanted an MPL placeholder:
  \begin{lstlisting}
  boost::mpl::placeholders::_1,
  boost::mpl::placeholders::_2,
  boost::mpl::placeholders::_3, ...
  \end{lstlisting}

  which gets tedious...
  \end{onlyenv}

  \begin{onlyenv}<2|handout:2>
    Lambda functions are a signature part of any functional programming language and also go very well
    with STL like algorithms.

    \vfill
    
    From our example earlier with \inline{square_f<Arg>}, that function in itself seems a bit redundant as it
    can easily be written as \inline{multiply<Arg,Arg>} with the same argument. But we run into two problems:

    \vfill

    \begin{itemize}
      \item The \inline{multiply} function is a metafunction, while the \inline{nest} function takes a metafunction class (a
        functor).
      \item We have no way of reducing \inline{multiply}'s argument list to only take one argument
    \end{itemize}

    \vfill

    MPL's placeholders solve this!
  \end{onlyenv}

  \begin{onlyenv}<3|handout:3>
  \begin{exampleblock}{With MPL lambda functions}
  \begin{lstlisting}[escapeinside={(*}{*)}]
template <class F, class X, unsigned N>
struct nest
  : nest<F, typename F::template apply<X>::type, N-1>
{};

template <class F, class X>
struct nest <F,X,0>
  : X
{};

int main(int, char**)
{
  typedef integer<5> five;
  nest<
    (*\tikzmark{lambda7}*)mpl::lambda< multiply<_1,_1> >::type(*\tikzmark{lambda8}*),five,3
  >::type::value;
}
  \end{lstlisting}
  \end{exampleblock}

  \begin{tikzpicture}[overlay, remember picture,decoration={snake,amplitude=.5pt,segment length=3pt,post length=0pt}]
    \draw[decorate, RoyalPurple] 
      ($ (lambda7) + (0,-.1cm)$) -- ( $ (lambda8) + (0,-.1cm) $);
  \end{tikzpicture}

  \end{onlyenv}

  \begin{onlyenv}<4|handout:4>
    \begin{exampleblock}{With {\inline{mpl}\texttt{::}\inline{apply}} {\color{OliveGreen} and placeholders}}
  \begin{lstlisting}[escapeinside={(*}{*)}]
template <class F, class X, unsigned N>
struct nest
  : nest<F, (*\tikzmark{lambda1}*)typename mpl::apply<F,X>::type(*\tikzmark{lambda2}*), N-1>
{};

template <class F, class X>
struct nest <F,X,0>
  : X
{};

int main(int, char**)
{
  typedef integer<5> five;
  (*\tikzmark{lambda3}*)nest<multiply<_1,_1>,five,3>::type::value(*\tikzmark{lambda4}*);
}
  \end{lstlisting}
  \end{exampleblock}

  \begin{tikzpicture}[overlay, remember picture]
    \draw[decoration={snake,amplitude=.5pt,segment length=3pt,post length=0pt},decorate, RoyalPurple] 
      ($ (lambda1) + (0,-.1cm)$) -- ( $ (lambda2) + (0,-.1cm) $);
    \draw[decoration={snake,amplitude=.5pt,segment length=3pt,post length=0pt},decorate, RoyalPurple] 
      ($ (lambda3) + (0,-.1cm)$) -- ( $ (lambda4) + (0,-.1cm) $);
  \end{tikzpicture}

  \end{onlyenv}

  \note<4|handout:4>
  {
    First option is to use \inline{mpl::lambda} to change the call to \inline{nest} without changing nest itself.
    As one can see, using the MPL lambda functions is very flexible and can easily be incorporated to work with
    existing metaprograms.

    \vspace{1em}

    The special \inline{mpl::apply} function scans the \inline{Func} type and turns it into a metafunction class
    if it contains any placeholders.
    
    \vspace{1em}

    Could of course use \inline{mpl::int\_<N>} and \inline{mpl::times<Arg1,Arg2>} throughout, but we will stick with
    our previously defined metadata and metafunctions to see that they are compatible with the rest of boost.
  }

  \vfill

\end{frame}

\begin{frame}[fragile]
  \frametitle{Control structures}

  Previously: Used template specialisation to switch between implementations

  \begin{exampleblock}{Simple template specialisation}
    \begin{lstlisting}[escapeinside={(*}{*)}]
template <class Type, bool FastImpl>
struct algorithm
{
  void operator() (const Type &)
  {
    // faster algorithm
  }
};

template <class Type>(*\tikzmark{safebegin}*)
struct algorithm<Type,false>
{
  void operator() (const Type &)
  {
    // safer algorithm
  }
};(*\tikzmark{safeend}*)
    \end{lstlisting}
  \end{exampleblock}

  \begin{tikzpicture}[overlay, remember picture]
    \coordinate (safeupper) at ($(safebegin) + (2.5cm,.6em)$);
    \draw[decorate,decoration=brace] (safeupper) -- (safeupper |- safeend)
      node[midway,right=.5cm,font=\footnotesize,align=center] {Specialised for \\ \lstinline!FastImpl = false!};
  \end{tikzpicture}

\end{frame}

\begin{frame}[fragile]
  \frametitle{Control structures}

  \vfill

  With TMP we can do more sophisticated checks and switches

  \vfill

  %\begin{exampleblock}{Simple template specialisation}
  \begin{onlyenv}<1|handout:1>
  \begin{exampleblock}{One more level of indirection}
  \begin{lstlisting}
struct fast_algorithm
{
  template <class Itt1, class Itt2>
  static void execute(Itt1, Itt2);
};

struct safe_algorithm
{
  template <class Itt1, class Itt2>
  static void execute(Itt1, Itt2);
};
  \end{lstlisting}
  \end{exampleblock}
  \end{onlyenv}

  \begin{onlyenv}<2|handout:2>
  \begin{exampleblock}{Choosing an implementation}
  \begin{lstlisting}
struct algorithm
{
  template <class Itt1, class Itt2>
  static void execute(Itt1 i1, Itt2 i2)
  {
    mpl::if_<
      typename mpl::and_<
        is_random_access<Itt1>,
        is_random_access<Itt2>
      >::type,
      fast_algorithm,
      safe_algorithm
    >::type::execute(i1,i2);
  }
};
  \end{lstlisting}
  \end{exampleblock}
  \end{onlyenv}

  \note<2|handout:2>
  {
    See that we have a unified interface that itself makes sure that it isn't misused and chooses the correct implementation
    depending on the type it is given.

    \vspace{1em}

    Example usage: could unify the STL sort algorithm. \inline{std::sort} only takes random access operators, while things such
    as \inline{std::set} and \inline{std::map} come pre sorted and \inline{std::list} have a sorting member function. Could
    write a function that takes arbitrary containers or iterators and use the correct implementation behind the scene. Drawback:
    hides the complexity of the operation.
  }

  \vfill

\end{frame}

\begin{frame}[fragile]
  \frametitle{Containers and iterators}

  \vfill

  boost provides a complete STL like container and algorithm library.

  \vfill

  Different containers have different access concepts

  \vfill

  \begin{center}
  \begin{tabular}{|l|} \hline
    Forward sequence \\
    {\lstinline[basicstyle=\ttfamily\normalsize]!begin<S>, end<S>, size<S>, front<S>!} \\
    {\lstinline[basicstyle=\ttfamily\normalsize]!push_front<S,x>, pop_front<S>!} \\
    {\lstinline[basicstyle=\ttfamily\normalsize]!insert<S,it,x>, erase<S,it>, clear<s>!} \\ \hline
    Bidirectional sequence \\
    {...\lstinline[basicstyle=\ttfamily\normalsize]!, back<S>, push_back<S,x>, pop_back<S>!} \\ \hline
    Random access sequence \\
    {...\lstinline[basicstyle=\ttfamily\normalsize]!, at<S,n>!} \\ \hline
  \end{tabular}
  \end{center}

  \vfill

  All functions return new sequences because we have no mutable objects.

  \vfill

\end{frame}

\begin{frame}[fragile]
  \frametitle{Containers and iterators: Short example}
  \framesubtitle{\inline{mpl::transform} and \inline{mpl::vector}}

  \vfill

  \vspace{1em}
  
  \begin{overprint}
  \begin{lstlisting}[escapeinside={(*}{*)}]
typedef mpl::vector<
  integer<3>, integer<7>, integer<-1> > my_vector;(*\tikzmark{vec1}*)

typedef mpl::transform<
  my_vector,
  multiply<_1,_1>
>::type square_vector;(*\tikzmark{vec2}*)

typedef mpl::begin<square_vector>::type begin;(*\tikzmark{vec3}*)
typedef mpl::next<begin>::type next;(*\tikzmark{vec4}*)

mpl::is_same<
  mpl::deref<next>::type,
  integer<49>
>::value;(*\tikzmark{vec5}*)
  \end{lstlisting}

  \only<1|handout:0>{
  \begin{center}
    \begin{tikzpicture}[every node/.style={inner sep=0pt},remember picture]
      \node[scale=1.5] (A1) {\{};
      \node[right=5pt of A1] (B1) {\strut$3,$};
      \node[right=5pt of B1] (C1) {\strut$7,$};
      \node[right=5pt of C1] (D1) {\strut$-1$};
      \node[scale=1.5, right=5pt of D1] {\}};
    \end{tikzpicture}
  \end{center}
  }

  \only<2-4|handout:0>{
  \begin{center}
    \begin{tikzpicture}[every node/.style={inner sep=0pt},remember picture]
      \node[scale=1.5] (A2) {\{};
      \node[right=5pt of A2] (B2) {\strut$9,$};
      \node[right=5pt of B2] (C2) {\strut$49,$};
      \node[right=5pt of C2] (D2) {\strut$1$};
      \node[scale=1.5, right=5pt of D2] {\}};
    \end{tikzpicture}
  \end{center}
  }

  \only<5|handout:0>{
  \begin{center}
    \tikz \node {\inline{true}};
  \end{center}
  }
  \end{overprint}

  \begin{tikzpicture}[overlay, remember picture]
    \coordinate (p1s) at ($(vec1) + (1em,.4ex)$);
    \coordinate (p1e) at ($(p1s) + (.5,0)$);
    \draw <1|handout:0> [<-,>=stealth] (p1s) -- (p1e) 
      node[right = 5pt, minimum size=5pt, inner sep=0pt, draw, circle, fill=Bittersweet] {};
    \coordinate (p2s) at ($(vec2) + (1em,.4ex)$);
    \coordinate (p2e) at (p2s -| p1e);
    \draw <2|handout:0> [<-,>=stealth] (p2s) -- (p2e) 
      node[right = 5pt, minimum size=5pt, inner sep=0pt, draw, circle, fill=Bittersweet] {};
    \coordinate (p3s) at ($(vec3) + (1em,.4ex)$);
    \coordinate (p3e) at (p3s -| p1e);
    \draw <3|handout:0> [<-,>=stealth] (p3s) -- (p3e) 
      node[right = 5pt, minimum size=5pt, inner sep=0pt, draw, circle, fill=Bittersweet] {};
    \coordinate (p4s) at ($(vec4) + (1em,.4ex)$);
    \coordinate (p4e) at (p4s -| p1e);
    \draw <4|handout:0> [<-,>=stealth] (p4s) -- (p4e) 
      node[right = 5pt, minimum size=5pt, inner sep=0pt, draw, circle, fill=Bittersweet] {};
    \coordinate (p5s) at ($(vec5) + (1em,.4ex)$);
    \coordinate (p5e) at (p5s -| p1e);
    \draw <5|handout:0> [<-,>=stealth] (p5s) -- (p5e) 
      node[right = 5pt, minimum size=5pt, inner sep=0pt, draw, circle, fill=Bittersweet] {};
  \end{tikzpicture}

  \begin{tikzpicture}[overlay, remember picture]
    \draw <3|handout:0> [<-,>=stealth] ($(B2) + (0,-1em)$) -- +(0,-1em);
    \draw <4|handout:0> [<-,>=stealth] ($(C2) + (0,-1em)$) -- +(0,-1em);
  \end{tikzpicture}

\end{frame}

\begin{frame}
  \frametitle{Where to go from here?}

  \vfill

  \begin{center}
    \tikz \node[font=\large,color=RoyalBlue] {\uline{Try it for yourself!}};
  \end{center}

  \vfill

  \begin{onlyenv}<2>
  \begin{itemize}
    \item Try to write simple programs
      \begin{itemize}
        \item Calculate an arithmetic sum
        \item Sum up the elements of a vector
        \item Implement your own for-loop
        \item ...
      \end{itemize}
    \item Study the literature
    \item Familiarise yourself with the boost MPL library
    \item See if you can make use of type switching in your own programs
    \item See if you can catch potential errors in your own programs
  \end{itemize}
  \end{onlyenv}

  \vfill

\end{frame}

\begin{frame}[fragile]
  \frametitle{Summary}

  \begin{itemize}
    \item We have seen how we can use the C++ template system to write metaprograms that look like normal programs.
    \item Metadata are types that contain their value in a public \inline{::value} type.
    \item Metafunctions are called by their public \inline{::type} type
      \begin{lstlisting}
  some_metafunction<Arg1,Arg2,...,ArgN>::type
      \end{lstlisting}
    \item Language facilitates a functional programming style with functions that manipulate other functions
    \item boost's MPL library implement a lot of useful metafuntions and types
  \end{itemize}

\end{frame}

\end{document}
