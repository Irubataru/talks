\newcommand{\dimtext}[2]%
{
  { \transparent{0.7}
  \begin{tikzpicture}[overlay, remember picture]
    \fill[white] ( #1 -| current page.north west) -- ++(0,.8em) -- ++(\paperwidth,0) -- (#2 -| current page.north east)
   -- ++(0,-.5em) -- ++(-\paperwidth,0) -- cycle;
  \end{tikzpicture}
  }
}

\newcommand{\specialcell}[2][c]{%
  \begin{tabular}
    [#1]{@{}c@{}}#2
  \end{tabular}
}

\newcommand{\tikzmark}[1]{\tikz[overlay,remember picture] \coordinate (#1);}

\newcommand{\qmat}{\ensuremath{Q_{\mathrm{uark}}}}
\newcommand{\mathfont}{\fontsize{10pt}{12pt}}
\newcommand{\minus}{\scalebox{0.75}[1.0]{$-$}}
\newcommand{\sectionframe}{%
  \addtocounter{framenumber}{-1}%
  \frame[plain]{\begin{center}\LARGE \color{beameralert} \insertsection\end{center}}}

\newcommand{\breakline}{%
  \begin{center} \begin{tikzpicture}
      \draw[beamerprimary] ({-0.5 * \textwidth},0) -- ({0.5 * \textwidth},0);
      \node[inner sep=0, minimum size=7pt, fill=white,circle] {};
      \node[inner sep=0, minimum size=4pt, draw=beamerprimary,circle] {};
      \node[inner sep=0, minimum size=1pt, fill=beamerprimary,circle] {};
  \end{tikzpicture} \end{center}
}
