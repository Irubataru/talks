\documentclass[14pt]{beamer}
\usepackage[T1]{fontenc}
\usepackage{url}
\usepackage{xcolor}
\usepackage{amsmath,varwidth}
\usepackage{tikz}
\usepackage{listings,algpseudocode}
\usepackage[quiet]{mathspec}
\usepackage{forest}

\setmainfont[
ItalicFont={Yanone Kaffeesatz Light},
Scale=1.3,
LetterSpace=2.0
]{Yanone Kaffeesatz Bold}

\setmonofont{Fira Code}

\newcommand{\BazelNavIcon}{%
\nointerlineskip
\begin{tikzpicture}[overlay,remember picture]
  \node[anchor=north east] at ([xshift=-1mm] current page.north east) {%
    \includegraphics[scale=0.07]{images/bazel-navicon.png}%
  };
\end{tikzpicture}
}

% --------------------- %
% Beamer theme settings %
% --------------------- %

\usetheme{CodeCourse}
\setbeamertemplate{navigation symbols}{} %Remove beamer navigation symbols

% ------------- %
% tikz settings %
% ------------- %

\usetikzlibrary{arrows.meta}

% ------------- %
% Font settings %
% ------------- %

% Load the eulervm package in the preamble to make sure one imports all the
% necessary mathematics symbols needed, then to get the font specification I
% want I need to lie to beamer about which font has which feature.

% Tells beamer that math mode should use serif fonts
% \usefonttheme[onlymath]{serif}

% Sets the serif font (which is the main font)
% \setmainfont[
%   ItalicFont={Linux Libertine O}
% ]{Euler Math}

% Sets the font beamer uses, which is sans-serif by default
% \setsansfont[
%   ItalicFont={Yanone Kaffeesatz Light},
%   Scale=1.3,
%   LetterSpace=2.0
% ]{Yanone Kaffeesatz Bold}

% Replaces every explicit \mathrm call to use this font
% \setmathrm{Linux Libertine O}

% -------------------------- %
% Standard LaTeX definitions %
% -------------------------- %

\DeclareGraphicsExtensions{.pdf}
\renewcommand\textbullet{\ensuremath{\bullet}}

% ----------------------------- %
% Include the other definitions %
% ----------------------------- %

% ------------------------ %
% Libraries I tend to need %
% ------------------------ %

\usetikzlibrary{calc,intersections,positioning,matrix,chains,scopes}
\usetikzlibrary{decorations.pathmorphing,decorations.pathreplacing,shapes}
\usetikzlibrary{decorations.text,backgrounds}
\usetikzlibrary{arrows.meta,bending}

\tikzset{
  pointy arrow/.style={
    {Stealth[flex]}-
  },
  tree-node/.style={
    draw=#1,
    fill=#1,
    text opacity=1,
    fill opacity=0.2,
    rounded corners=1
  },
  expression/.style={
    tree-node = FeebleWeek
  },
  operator/.style={
    tree-node = Marty
  },
  expr-leaf/.style={
    tree-node = UnrealFoodPills,
  }
}

% --------------------------- %
% The styles I use for coding %
% --------------------------- %

\lstset{%
  language=[11]C++,
  basicstyle=\ttfamily\lst@ifdisplaystyle\fontsize{9pt}{9pt}\selectfont\fi,
  keywordstyle=\color{sgreen}, identifierstyle=\color{sblue}, 
  commentstyle=\color{sbase1}, stringstyle=\color{sorange},
  numberstyle=\color{sviolet}, showstringspaces=false,  
  breaklines=true,
  tabsize=5,
  escapeinside={(*}{*)},
  morekeywords={%
    unique_ptr,shared_ptr,weak_ptr,
    make_unique, make_shared,
    list,vector,map,stack,queue,array,
    Widget,constexpr,decltype,variant,optional,string,
    is_same,value,is_same_v,type,value
  }
}

\lstalias[]{gnumake}[gnu]{make}

\def\inline{\lstinline[basicstyle=\ttfamily\normalsize]}

% ---------------------------------------------------------------------------------------------------- %
% Various command for inline code instead of lstinline so that I don't have to update the keyword list %
% ---------------------------------------------------------------------------------------------------- %

\newcommand{\code}[1]{\color{black}#1}
\newcommand{\object}[1]{{\color{sgreen}#1}}
\newcommand{\function}[1]{{\color{sblue}#1}}
\newcommand{\namespace}[1]{{\color{sblue}#1}}

\newcommand{\std}[1]{{\code{\namespace{std}::#1}}}
\newcommand{\boost}[1]{{\code{\namespace{boost}::#1}}}

\newcommand{\uniqueptr}{\std{unique\_ptr}}
\newcommand{\sharedptr}{\std{shared\_ptr}}
\newcommand{\weakptr}{\std{weak\_ptr}}
\newcommand{\autoptr}{\std{auto\_ptr}}



\usefonttheme{serif}

% Turn off page numbering
\setbeamertemplate{footline}{}

\title{Introduction to template metaprogramming}
\author{\texorpdfstring{%
    Aleksandra Rylund Glesaaen\newline%
    \fontsize{12pt}{12pt}\selectfont\texttt{aleksandra@glesaaen.com}%
  }{%
    Aleksandra Rylund Glesaaen}}

\date{\texorpdfstring{%
    SA{\color{SA2COrange}\textsuperscript{2}}C Seminar Series\newline{}February 23rd 2018%
  }{%
    February 23rd 2018}}

\begin{document}

\nocite{*}

\frame{\titlepage}

\begin{frame}{Talk outline}

  \begin{enumerate} \setlength\itemsep{.5em}
    \item The basics, metaprogramming vocabulary
    \item Tools and libraries
    \item Expression templates
  \end{enumerate}
  
\end{frame}

\section{The basics}
\frame{\sectionpage}

\begin{frame}[fragile]
  \frametitle{Vocabulary: metadata}

  A constant "value" accessible by calling \code{::\namespace{value}}

  \begin{overlayarea}{\textwidth}{4cm}
  \begin{onlyenv}<1>
  \begin{lstlisting}[basicstyle=\codefontsize{12pt}]
struct a_value
{
  constexpr int value = ...;
};
  \end{lstlisting}
  \end{onlyenv}

  \begin{onlyenv}<2>
  \begin{lstlisting}[basicstyle=\codefontsize{12pt}]
struct a_value
{
  using type = a_value;(*\tmark{self-ref}*)
  constexpr int value = ...;
};
  \end{lstlisting}
  \end{onlyenv}

  \begin{lstlisting}[basicstyle=\codefontsize{12pt}]
a_value::value;
  \end{lstlisting}

  \only<1>{
    \vspace{.5cm}

    Made easier with C++14 template variables\\
    {\itshape\changefontsize{10pt}(we will not use these here though, for a reason)}
  }
  \end{overlayarea}

  \begin{onlyenv}<2>
    \nointerlineskip
    \begin{tikzpicture}[overlay,remember picture]
      \draw[{Stealth}-] ([shift={(1mm,1mm)}] self-ref) -- +(2cm,0)
        node[right,scale=0.8,align=left,yshift=-3mm] {Usually self\\referential};
    \end{tikzpicture}
  \end{onlyenv}

\end{frame}

\begin{frame}[fragile]
  \frametitle{Vocabulary: metafunction}

  A "function" that takes its arguments through templates and stores the result
  in \code{::\namespace{type}}

  \vspace{.5cm}

  \begin{lstlisting}[basicstyle=\codefontsize{12pt}]
struct a_function
{
  template <int x, int y>
  using type = ...;
};

a_function<5, 6>::type;
  \end{lstlisting}

\end{frame}

\begin{frame}[fragile]
  \frametitle{Vocabulary: metafunction class}

  A function object that can itself be treated as a type. Function call accessed
  by a nested metafunction named \code{::\namespace{apply}}

  \vspace{.25cm}

  \begin{lstlisting}[basicstyle=\codefontsize{12pt}]
struct a_metafunction_class
{
  template <typename Arg1, typename Arg2>
  struct apply
  {
    ...
  };
};
  \end{lstlisting}

\end{frame}

\begin{frame}[fragile]
  \frametitle{Example: Simple algebra}

  \begin{lstlisting}[basicstyle=\codefontsize{9pt}]
template <int N>
struct integer
{
  using type = integer<N>;
  static constexpr int value = N;(*\tmark{metadata}*)
};
  \end{lstlisting}

  \begin{lstlisting}[
    basicstyle=\codefontsize{9pt},
    morekeywords={integer}]
template <typename Arg1, typename Arg2>
struct multiply
{
  using type =(*\tmark{metafunction}*)
    integer<Arg1::value * Arg2::value>;
};
  \end{lstlisting}

  \begin{lstlisting}[
    basicstyle=\codefontsize{9pt},
    morekeywords={integer,multiply}]
using five = integer<5>;
using m_nine = integer<-9>;

constexpr auto result = 
  multiply<five, m_nine>::type::value;
  \end{lstlisting}

  \begin{onlyenv}<2>
    \nointerlineskip
    \begin{tikzpicture}[overlay,remember picture]
      \draw[pointy arrow]
        ([shift={(1mm,1mm)}] metadata) -- +(1.75cm,0)
        node[right,scale=0.8] (label-1) {metadata};
      \draw[pointy arrow,transform canvas={yshift=1mm}] 
        ([xshift=3mm] metafunction) -- (metafunction -| label-1.west)
        node[right,scale=0.8] {metafunction};
    \end{tikzpicture}
  \end{onlyenv}

\end{frame}

\begin{frame}[fragile]
  \frametitle{Example: Simple algebra}

  \begin{lstlisting}[basicstyle=\codefontsize{9pt}]
template <int N>
struct integer {...};
  \end{lstlisting}

  \begin{lstlisting}[basicstyle=\codefontsize{9pt}, morekeywords={integer}]
template <typename Arg1, typename Arg2>
struct multiply
  : integer< Arg1::value * Arg2::value > {};(*\tmark{forwarding}*)
  \end{lstlisting}

  \begin{lstlisting}[basicstyle=\codefontsize{9pt}, morekeywords={multiply}]
struct square_f
{
  template <typename Arg>
  struct apply(*\tmark{metafunction-class}*)
    : multiply<Arg, Arg>
  {};
};
  \end{lstlisting}

  \begin{lstlisting}[basicstyle=\codefontsize{9pt}, morekeywords={square_f}]
square_f::apply<five>::value;
  \end{lstlisting}

  \begin{onlyenv}<2>
    \nointerlineskip
    \begin{tikzpicture}[overlay,remember picture]
      \draw[pointy arrow]
        ([shift={(-1cm,4mm)}] forwarding) 
        .. controls +(2mm,2mm) and +(0,-2mm) ..
        +(7mm,1cm)
        node[above,scale=0.8] (label-1) {metafunction forwarding};
      \draw[pointy arrow, transform canvas={yshift=1mm}]
        ([xshift=5mm] metafunction-class) -- (metafunction-class -| label-1.west)
        node[right,scale=0.8] {metafunction class};
    \end{tikzpicture}
  \end{onlyenv}
  
\end{frame}

\begin{frame}[fragile]
  \frametitle{Example: Functors}

  Template metaprogramming has no variables and is thus inherently a functional
  programming language.

  \vspace{.5cm}
  
  Let us implement something functional, like nest:

  \vspace{.5cm}

  \begin{center}
    \code{%
      \function{nest}(\function{f}, x, 5) %
        = \function{f}(\function{f}(\function{f}(\function{f}(\function{f}(x)))))%
    }
  \end{center}

\end{frame}

\begin{frame}[fragile]
  \frametitle{Example: Nest function}
  
  \begin{lstlisting}[basicstyle=\codefontsize{10pt}]
template <typename F, typename X, unsigned N>
struct nest
    : (*\tmark{nest-nest}*)nest<F,
           (*\tmark{u-begin}*)typename F::template apply<X>::type(*\tmark{u-end}*),
           N - 1> {};
  \end{lstlisting}

  \vspace{.25cm}

  \begin{lstlisting}[basicstyle=\codefontsize{10pt}]
template <typename F, typename X>
struct nest<F, X, 0>(*\tmark{nest-end}*)
  : X {};
  \end{lstlisting}

  \vspace{.25cm}

  \begin{lstlisting}[
    basicstyle=\codefontsize{10pt},
    morekeywords={nest,squared_f}]
using five = integer<5>;
nest<squared_f, five, 3>::type::value;(*\tmark{result}*)
  \end{lstlisting}

  \begin{onlyenv}<2>
    \nointerlineskip
    \begin{tikzpicture}[overlay,remember picture]
      \underlinemark[sorange,opacity=0.5]{u-begin}{u-end}

      \draw[{Stealth}-,rounded corners]
        ([shift={(5mm,-1mm)}] nest-nest) |- +(5.5cm,-1.2cm)
        node[right,scale=0.7] (recursion) {recursion};

      \draw[{Stealth}-,transform canvas={yshift=1mm}]
        ([xshift=1mm] nest-end) -- (nest-end -| recursion.west)
        node[right,scale=0.7] {end of recursion};

      \node at ([shift={(.5cm,1mm)}] result) [right, scale=0.8]
        {$ = ((5^2)^2)^2$};

    \end{tikzpicture}
  \end{onlyenv}

\end{frame}

\begin{frame}[fragile]
  \frametitle{Example: Nest function}

  \begin{lstlisting}[
    basicstyle=\codefontsize{11pt},
    morekeywords={nest,square_f}]
int main()
{
  using five = integer<5>;
  using result
    = nest<square_f, five, 3>::type;

  return result::value;
(*\tmark{begin-code}*)}
  \end{lstlisting}

  \vspace{4cm}
  
  \lstset{language=[x86masm]Assembler}
  \nointerlineskip
  \begin{tikzpicture}[overlay,remember picture]

    \coordinate (code-left-edge) at (current page.west |- begin-code);
    \coordinate (code-right-edge) at (current page.east |- begin-code);

    \draw<2->[-{Stealth}, transform canvas={xshift=2cm}]
      ([yshift=-2mm] code-left-edge) -- +(0,-1cm)
      node[midway, right, scale=0.8]
      {Compile, -O0};

    \draw<3>[-{Stealth}, transform canvas={xshift=7.1cm}]
      ([yshift=-2mm] code-left-edge) -- +(0,-1cm)
      node[midway, right, scale=0.8]
      {Compile, -O1};
    
    \begin{scope}[transform canvas={yshift=-1.5cm}]
      \fill<2-> [FeebleWeek,opacity=0.2] 
        (code-left-edge) rectangle (current page.south east);
    
      \begin{onlyenv}<2->
      \node[anchor=north west] at ([shift={(-2mm,-.5mm)}] begin-code) (comp-1) {%
      \begin{lstlisting}[basicstyle=\codefontsize{10pt}]
main:
  push rbp
  mov rbp, rsp
  mov eax, 390625
  pop rbp
  ret
      \end{lstlisting}};
      \end{onlyenv}

      \begin{onlyenv}<3>
      \node[anchor=north west] at ([xshift=1cm] comp-1.north east) {%
      \begin{lstlisting}[basicstyle=\codefontsize{10pt}]
main:
  mov eax, 390625
  ret
      \end{lstlisting}};
      \end{onlyenv}
    \end{scope}
  \end{tikzpicture}
  \lstset{language=[11]C++}

  
\end{frame}

\section{Tools and libraries}
\frame{\sectionpage}

\begin{frame}
  \frametitle{\namespace{brigand}}

  \namespace{brigand} is a light-weight C++11 reimagining of
  \boost{\namespace{mpl}}, a metaprogramming library

  \vspace{.5cm}

  \uncover<2>{
  Features:

  \begin{itemize}
    \item Functional programming utilities
    \item Sequence storage
    \item Sequence algorithms
    \item Arithmetic operations
    \item ...and more
  \end{itemize}
  }

\end{frame}

\begin{frame}[fragile]
  \frametitle{Simplifying nest with \namespace{brigand}}
  
  \begin{lstlisting}[basicstyle=\codefontsize{10pt}]
#include <brigand/brigand.hpp>
using brigand::_1;

template <typename F, typename X, unsigned N>
struct nest
    : nest<F,
           (*\tmark{u-1-begin}*)typename brigand::apply<F, X>::type(*\tmark{u-1-end}*),
           N - 1>
{};
  \end{lstlisting}

  \begin{lstlisting}[
    basicstyle=\codefontsize{10pt},
    morekeywords={integer,nest,multiply}]
int main()
{
  using five = integer<5>;
  return nest<(*\tmark{u-2-begin}*)multiply<_1, _1>(*\tmark{u-2-end}*), five, 3>::value;
}
  \end{lstlisting}

  \begin{onlyenv}<2>
    \nointerlineskip
    \begin{tikzpicture}[overlay,remember picture]
      \underlinemark[sorange,opacity=0.5,transform canvas={yshift=.5pt}]{u-1-begin}{u-1-end}
      \underlinemark[sorange,opacity=0.5]{u-2-begin}{u-2-end}
    \end{tikzpicture}
  \end{onlyenv}
\end{frame}

\begin{frame}[fragile]
  \frametitle{Sequences at compile time}

  \begin{lstlisting}[
    basicstyle=\codefontsize{10pt},
    morekeywords={sort,less,sizeof_}
    ]
#include <brigand/brigand.hpp>
using brigand::_1;
using brigand::_2;

using type_list =
    brigand::list<double, short, char, int>;

using sorted_list = brigand::sort<
    type_list,
    brigand::less<brigand::sizeof_<_1>,
                  brigand::sizeof_<_2>>>;
  \end{lstlisting}
  
\end{frame}

\begin{frame}
  \frametitle{\namespace{metal}}

  Provides much the same functionality as \namespace{brigand}

  \vspace{.5cm}

  \uncover<2>{
  Features:

  \begin{itemize}
    \item Lambda calculus
    \item Sequence storage
    \item Sequence transformations
    \item Typemaps
    \item ...and more
  \end{itemize}
  }

\end{frame}

\begin{frame}[fragile]
  \frametitle{Simplifying nest with \namespace{metal}}
  
  \begin{lstlisting}[
    basicstyle=\codefontsize{10pt},
    morekeywords={invoke}
    ]
#include <metal/metal.hpp>
using metal::_1;

template <typename F, typename X, unsigned N>
struct nest
    : nest<F,
           typename metal::invoke<F, X>::type,
           N - 1>
{};
  \end{lstlisting}

  \begin{lstlisting}[
    basicstyle=\codefontsize{10pt},
    morekeywords={bind,lazy,nest,integer}
    ]
int main()
{
  using five = integer<5>;
  return nest<
      metal::bind<metal::lazy<multiply>, _1, _1>,
      five, 3>::value;
}
  \end{lstlisting}
  
\end{frame}

\begin{frame}
  \frametitle{\object{constexpr}}

  A directive for telling the compiler that the expression is computable at
  compile time

  \vspace{.5cm}
  C++11/C++14:
  
  \begin{itemize}
    \item \object{constexpr} values
    \item \object{constexpr} functions
  \end{itemize}

  \vspace{.5cm}
  C++17:
  
  \begin{itemize}
    \item \object{constexpr} ifs
    \item \object{constexpr} lambdas
  \end{itemize}


\end{frame}

\begin{frame}[fragile]
  \frametitle{nest with \object{constexpr} C++14}

  \begin{lstlisting}[
    basicstyle=\codefontsize{10pt},
    morekeywords={forward}]
constexpr long square(long x) { return x * x; }

template <typename Function>
constexpr auto nest(Function &&f, long val,
                    long N)
{
  if (N == 0)
    return val;
  else
    return nest(std::forward<Function>(f),
                f(val),
                N - 1);
}

static_assert(nest(square, 5, 3) == 390625, "");
  \end{lstlisting}
  
\end{frame}

\begin{frame}[fragile]
  \frametitle{nest with \object{constexpr} C++17}

  \begin{lstlisting}[
    basicstyle=\codefontsize{10pt},
    morekeywords={forward}]
template <typename Function>
constexpr auto nest(Function &&f, long val,
                    long N)
{
  if (N == 0)
    return val;
  else
    return nest(std::forward<Function>(f),
                f(val),
                N - 1);
}

static_assert(
  nest([](auto x) { return x * x; }, 5, 3)
    == 390625);
  \end{lstlisting}
  
\end{frame}

\begin{frame}
  \frametitle{What is \object{constexpr}}

  \begin{itemize} \setlength\itemsep{.5em}
    \item Can only call other \object{constexpr} values
    \item ... or \object{constexpr} functions
    \item ... or \object{constexpr} classes
    \begin{itemize}
      \item these need a \object{constexpr} constructor
    \end{itemize}
  \end{itemize}

\end{frame}

\begin{frame}[fragile]
  \frametitle{\object{constexpr} if}
  
  \begin{onlyenv}<1>
  \begin{lstlisting}[basicstyle=\codefontsize{12pt}]
void only_int(int) {}

template <typename T>
void will_fail(T x)
{
  if (std::is_same_v<T, int>)
    only_int(x);
}

int main()
{
  will_fail(std::string{});(*\tmark{f-call}*)
}
  \end{lstlisting}

  \end{onlyenv}

  \begin{onlyenv}<2->
  \begin{lstlisting}[basicstyle=\codefontsize{12pt}]
void only_int(int) {}

template <typename T>
void wont_fail(T x)
{
  if constexpr(std::is_same_v<T, int>)
    only_int(x);(*\tmark{faulty-branch}*)
}

int main()
{
  wont_fail(std::string{});(*\tmark{f-call-2}*)
}
  \end{lstlisting}
  \end{onlyenv}

  \nointerlineskip
  \begin{tikzpicture}[overlay,remember picture]
    \draw<1>[{Stealth}-] ([shift={(1mm,1mm)}] f-call) -- +(1.5cm,0)
      node[right, scale=0.8, font=\color{sred}] {Compile error};
    \draw<2->[{Stealth}-] ([shift={(1mm,1mm)}] f-call-2) -- +(1.5cm,0)
      node[right, scale=0.8, font=\color{sgreen}] (compiles) {Compiles};

    \node<3> at (faulty-branch -| compiles.east)
      [right, scale=0.8, anchor=east, yshift=-10mm] (falsey-node)
      {falsey branch not compiled};
    \draw<3>[{Stealth}-] ([shift={(-4mm,-2mm)}] faulty-branch)
      .. controls +(2mm, -4mm) and +(-5mm,0) ..
      (falsey-node.west);
  \end{tikzpicture}

\end{frame}

\section{Expression templates: a first look}

% Custom sectionframe so that I can break the section name
\begin{frame}
  \usebeamercolor{section page background canvas}
  \begin{tikzpicture}[overlay,remember picture]
    \fill[bg] (current page.south west) rectangle (current page.north east);
  \end{tikzpicture}
  \centering
  \vskip1cm\par
  \begin{beamercolorbox}[sep=12pt,center]{section title}
    \usebeamerfont{section title}Expression templates:\\a first look\par
  \end{beamercolorbox}
\end{frame}

\begin{frame}
  \frametitle{The AST (Abstract Syntax Tree)}

  Similar to creating lazy evaluation expressions we want to create an AST, then
  hook into it and determine what it does when evaluated
  
\end{frame}

\begin{frame}[fragile,plain]

  \lstset{literate={+}{ + }1 {-}{ - }1 {*}{ * }1 {/}{ / }1}
  \begin{center}
  \begin{tikzpicture}[
    level distance=3cm,
    scale=0.8,every node/.style={scale=0.8},
    every node/.append style={
      text height=1.5ex,
      text depth=.5ex,
      minimum width=2em, align=center},
  ]
    \input{input/expr_tree.tex}
  \end{tikzpicture}
  \end{center}
  \lstset{literate=}

\end{frame}

\begin{frame}[fragile,plain]

  \lstset{literate={+}{ + }1 {-}{ - }1 {*}{ * }1 {/}{ / }1,linewidth=100cm}
  \begin{center}
  \begin{tikzpicture}[
    level distance=4cm,
    scale=0.28,every node/.style={scale=0.28},
    every node/.append style={
      text height=1.5ex,
      text depth=.5ex,
      minimum width=2em, align=center},
  ]
    \input{input/expr_tree_large.tex}
  \end{tikzpicture}
  \end{center}
  \lstset{literate=,linewidth=\linewidth}
  
\end{frame}

\begin{frame}[fragile,plain]

  \nointerlineskip
  \begin{tikzpicture}[overlay,remember picture]
    \node at (current page.center) {
      \begin{lstlisting}[basicstyle=\codefontsize{11pt}]
ofs << "\\"
    << (3_v + "x"_v / ("y"_v - 8_v) -
        (3_v * (1_v + "z"_v) / "w"_v) /
            (2_v - 6_v * "x"_v))
           .to_tex()
    << ";\n";
      \end{lstlisting}};
  \end{tikzpicture}
  
\end{frame}

\begin{frame}[fragile]
  \frametitle{First attempt}

  \begin{lstlisting}[
    basicstyle=\codefontsize{10pt},
    morekeywords={integer}]
template <typename Arg1, typename Arg2>
struct plus_expr
  : integer<Arg1::value + Arg2::value>
{};
  \end{lstlisting}

  \begin{lstlisting}[
    basicstyle=\codefontsize{10pt},
    morekeywords={integer,plus_expr}]
template <long N, long M>
auto operator+(integer<N>, integer<M>)
{
  return plus_expr<integer<N>, integer<M>>{};
}

int main()
{
  auto expr = five{}+eight{};

  std::cout << expr.value << std::endl;
}
  \end{lstlisting}

\end{frame}

\begin{frame}[fragile]
  \frametitle{Some obstacles}
  
  It doesn't do complicated expressions:
  \begin{lstlisting}[basicstyle=\codefontsize{14pt}]
  five{} + eight{} + five{};
  \end{lstlisting}

  \vspace{.2cm}

  \begin{overlayarea}{\textwidth}{5cm}
  \begin{onlyenv}<2-3>
  Possible solution:
  \begin{lstlisting}[
    basicstyle=\codefontsize{8pt},
    morekeywords={integer,plus_expr}]
template <long N, typename Arg1, typename Arg2>
auto operator+(integer<N>, plus_expr<Arg1, Arg2>)
{
  return plus_expr<integer<N>, plus_expr<Arg1, Arg2>>{};
}

template <long N, typename Arg1, typename Arg2>
auto operator+(plus_expr<Arg1, Arg2>, integer<N>)
{
  return plus_expr<plus_expr<Arg1, Arg2>, integer<N>>{};
}
  \end{lstlisting}
  \end{onlyenv}

  \begin{onlyenv}<4->
  Possible solution:
  \begin{lstlisting}[
    basicstyle=\codefontsize{10pt},
    morekeywords={plus_expr}]
template <typename Arg1, typename Arg2>
auto operator+(Arg1, Arg2)
{
  return plus_expr<Arg1, Arg2>{};
}
  \end{lstlisting}
  \end{onlyenv}

  \end{overlayarea}

  \nointerlineskip
  \begin{tikzpicture}[overlay,remember picture]
    \fill<3,5>[WhiteTrash, opacity=0.75] (current page.north west) rectangle (current page.south east);
    \node<3>[font=\color{Marty},scale=5, rotate=30] at (current page.center) {Tedious};
    \node<5>[font=\color{Marty},scale=3, rotate=30] at (current page.center) {Matches everything};
  \end{tikzpicture}

\end{frame}

\begin{frame}[fragile]
  \frametitle{Some obstacles}

  No obvious way to extend to new operators and types

  \begin{lstlisting}[basicstyle=\codefontsize{14pt}]
eight{} - five{} / eight{};
  \end{lstlisting}

  \vspace{.2cm}

  "Possible solution":
  \begin{lstlisting}[
    basicstyle=\codefontsize{7pt},
    morekeywords={plus_expr,minus_expr,times_expr,divides_expr}]
template <typename Arg1, typename Arg2, typename Arg3, typename Arg4>
auto operator+(plus_expr<Arg1, Arg2>, minus_expr<Arg3, Arg4>);

template <typename Arg1, typename Arg2, typename Arg3, typename Arg4>
auto operator/(minus_expr<Arg1, Arg2>, times_expr<Arg3, Arg4>);

template <typename Arg1, typename Arg2, typename Arg3, typename Arg4>
auto operator*(divide_expr<Arg1, Arg2>, plus_expr<Arg3, Arg4>);
  \end{lstlisting}

  \vspace{.2cm}

  Number of combinations $\sim$ (number of operations)\textsuperscript{3}

\end{frame}

\section{The Curiously Recurring Template Pattern}
\EmphFrame[Marty]{The time has finally come}
\frame{\sectionpage}

\begin{frame}[fragile]
  \frametitle{The CRTP}

  \begin{overlayarea}{\textwidth}{4.5cm}

  \begin{onlyenv}<1>
  \begin{lstlisting}[basicstyle=\codefontsize{11pt}]
template <typename T>
class Base
{};
  \end{lstlisting}
  \end{onlyenv}

  \begin{onlyenv}<2->
  \begin{lstlisting}[
    basicstyle=\codefontsize{11pt},
    escapeinside={[*}{*]}]
template <typename T>
class Base
{
  void base_call()
  {
    static_cast[*\tmark{static-cast}*]<T&>(*this).derived_call();
  }
};
  \end{lstlisting}
  \end{onlyenv}

  \end{overlayarea}

  \begin{overlayarea}{\textwidth}{4cm}

  \begin{onlyenv}<1>
  \begin{lstlisting}[
    basicstyle=\codefontsize{11pt},
    morekeywords={Base}]
class Derived : public Base<Derived>
{};
  \end{lstlisting}
  \end{onlyenv}

  \begin{onlyenv}<2->
  \begin{lstlisting}[
    basicstyle=\codefontsize{11pt},
    morekeywords={Base}]
class Derived : public Base<Derived>
{
  void derived_call();
};
  \end{lstlisting}
  \end{onlyenv}

  \end{overlayarea}

  \nointerlineskip
  \begin{tikzpicture}[overlay,remember picture]
    \draw<3> [{Stealth}-]
      ([shift={(-1cm, -2mm)}] static-cast)
      .. controls +(0,-.4cm) and +(-.8cm,0) ..
      +(1cm, -.7cm) node[right, scale=0.8] {\object{static\_cast}: no vtable lookup};
  \end{tikzpicture}

\end{frame}

\begin{frame}[fragile]
  \frametitle{Second attempt, pure types}

  \begin{lstlisting}[basicstyle=\codefontsize{10pt}]
template <typename T>
struct base_expr
{ };
  \end{lstlisting}

  \begin{lstlisting}[
    basicstyle=\codefontsize{10pt},
    morekeywords={base_expr}]
template <long N>
struct integer : base_expr<integer<N>>
{ ... };

template <typename Arg1, typename Arg2>
struct plus_expr
    : base_expr<plus_expr<Arg1, Arg2>>
{
  using type = 
    integer<Arg1::value + Arg2::value>;

  static constexpr long value = type::value;
};
  \end{lstlisting}

\end{frame}

\begin{frame}[fragile]
  \frametitle{Second attempt, pure types}

  \begin{lstlisting}[
    basicstyle=\codefontsize{10pt},
    morekeywords={base_expr,plus_expr}]
template <typename Arg1, typename Arg2>
constexpr auto operator+(base_expr<Arg1>,
                         base_expr<Arg2>(*\tmark{plus-args}*))
    -> plus_expr<Arg1, Arg2>
{
  return {};
}

static_assert(decltype(five{} + eight{} +
                       five{})::value == 18);
  \end{lstlisting}

  \uncover<3> {
    \vspace{.3cm}
    Doesn't actually use any of the CRTP functionality\\
    \object{base\_expr} is just a tag
  }

  \nointerlineskip
  \begin{tikzpicture}[overlay,remember picture]
    \draw<2->[{Stealth}-]
      ([shift={(-1.5cm,-2mm)}] plus-args)
      .. controls +(0,-.3cm) and +(-.8cm,0) ..
      +(1cm,-.5cm)
      node[right,scale=0.8,align=left,yshift=-3mm] {limits pattern\\matching};
  \end{tikzpicture}

\end{frame}

\begin{frame}[fragile]
  \frametitle{Second attempt, using the CRTP}

  \begin{lstlisting}[
    basicstyle=\codefontsize{10pt},
    escapeinside={[*}{*]}]
template <typename T>
struct base_expr
{
  constexpr auto value() const
  {
    return static_cast<T const&>(*this).value();
  }
};
  \end{lstlisting}

  \begin{lstlisting}[
    basicstyle=\codefontsize{10pt},
    morekeywords={base_expr}]
template <long N>
struct integer : base_expr<integer<N>>
{
  static constexpr long value()
  {
    return N;
  }
};
  \end{lstlisting}

\end{frame}

\begin{frame}[fragile]
  \frametitle{Second attempt, using the CRTP}

  \begin{lstlisting}[
    basicstyle=\codefontsize{9pt},
    morekeywords={base_expr}]
template <typename Arg1, typename Arg2>
struct plus_expr
    : base_expr<plus_expr<Arg1, Arg2>>
{
  constexpr plus_expr(base_expr<Arg1> const &le,
                      base_expr<Arg2> const &re)
      : le_{le}, re_{re} {};

  constexpr auto value() const
  {
    return le_.value() + re_.value();
  }

private:
  base_expr<Arg1> const &le_;
  base_expr<Arg2> const &re_;
};
  \end{lstlisting}

\end{frame}

\begin{frame}[fragile]
  \frametitle{Second attempt, using the CRTP}

  \begin{lstlisting}[
    basicstyle=\codefontsize{10pt},
    morekeywords={base_expr,plus_expr}]
template <typename Arg1, typename Arg2>
constexpr auto
operator+(base_expr<Arg1> const &le,
          base_expr<Arg2> const &re)
{
  return plus_expr<Arg1, Arg2>(le, re);
}

static_assert(
    (five{} + eight{} + five{}).value() == 18);
  \end{lstlisting}

  \uncover<2>{
    \vspace{.5cm}
    Simple :)
  }

\end{frame}

\begin{frame}
  \frametitle{Resources}

  \bibliographystyle{plain}
  {\footnotesize
    \bibliography{ref}}
\end{frame}

\begin{frame}
  \begin{center}
    {\changefontsize{28pt}\color{Tropiteal}Thanks!}
    \vspace{1cm}

    \begin{tikzpicture}
      \node[scale=0.075] (cat) {\includegraphics{figures/Octocat.png}};
      \node[right of=cat, anchor=west, font=\changefontsize{18pt}] {Irubataru/talks};
    \end{tikzpicture}
  \end{center}
\end{frame}

\end{document}
