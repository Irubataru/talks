\section{Complex Langevin and Random Matrix Theory}
\frame[plain]{\sectionpage}
\addtocounter{framenumber}{-1}

\begin{frame}
  \frametitle{Stochastic quantisation}

  \vskip .3cm
  Introduce a fictional flow time $\scalemath{1.3}{\alpha}$
  \[
    \phi_i(x) \to \phi_i(x, \alpha)
  \]
  and the stochastic flow in $\scalemath{1.3}{\alpha}$
  \[
    \frac{\partial \phi_i (x, \alpha)}{\partial \alpha} = - \frac{\delta
      S[\phi_i]}{\delta \phi_i} + \eta (x, \alpha)\tikzmark{noise}
  \]
  then if\; \raisebox{.02cm}{$\scalemath{0.8}{\frac{\delta S}{\delta \phi}} \in \mathbb{R}$}\hskip .2cm
  \inlinecite{Damgaard and Huffel, 1987}
  \[
    \lim_{\alpha \to \infty} \big\langle O [\phi(x, \alpha)]_i \big\rangle_{\eta} =
      \big\langle O [\phi(x)]_i \big\rangle
  \]

  \nointerlineskip
  \begin{tikzpicture}[overlay,remember picture]
    \only<2>{
      \draw[{Stealth[round]}-,line width=1pt,Marty]
        ([shift={(.2cm,.5ex)}] noise) .. controls +(.75,0) and +(0,-.75) ..  +(1,1)
          node[above, scale=0.8, align=center] {Stochastic\\noise};
    }
  \end{tikzpicture}
\end{frame}

\begin{frame}
  \frametitle{Complex Langevin}

  Much the same as Real Langevin, however

  \vskip .5cm
  \[
    \frac{\delta S}{\delta \phi} \in \raisebox{-.05cm}{$\mathbb{C}$}
    \hskip .5cm\Rightarrow\hskip .5cm \phi_i(x, \alpha) \in\raisebox{-.05cm}{$\mathbb{C}$}
  \]

  \vskip .5cm
  proof for convergence to the right theory no longer holds,
  exceptions have been found \inlinecite{Ambj\o{}rn and Yang, 1985}
  
\end{frame}

% Presentation note:
% Bring up example of times when people have set out to disprove something
% only to come up with an algorithm that fixes everything

\begin{frame}
  \frametitle{Why stochastic quantisation?}

  How the CL converges to the wrong theory not well understood

  \vskip .5cm
  Much interest in methods for "{\color{TealDrop}repairing}" CL

  \begin{itemize}
    \setlength\itemsep{.5em}
    \item Gauge cooling
    \item Utilising the Lefschetz thimbles
    \item ... ?
  \end{itemize}

  \vskip .35cm
  Need a signal for {\color{Marty}failure} applicable to QCD
  
\end{frame}

\begin{frame}
  \frametitle{Random matrix theory}

  Start with a simpler system that also suffer from a strong sign problem

  \vskip .75cm
  Same flavour symmetries as QCD (in the $\scalemath{1.3}{\epsilon}$-regime)

  \vskip .75cm
  Numerically a lot cheaper than full lattice studies

  \vskip .75cm
  First attempts studying the Osborn model\\
  \inlinecite{Mollgaard and Splittorff, 2013-2014; Nagata, Nishimura and
    Shimasaki, 2015-2016}
  
\end{frame}

\begin{frame}
  \frametitle{The Stephanov model}
  \framesubtitle{Definition \inlinecite{Stephanov, 1996}}

  \vskip -.5cm
  \begin{equation*}
    \mathcal{Z} = \int \big[d W\big] \, e^{- N \Sigma^2 \tr W^{\dagger} W}
      \det{}^{N_f}
      \begin{pmatrix}
        M & i W + \mu \\
        i W^{\dagger} + \mu & M
      \end{pmatrix}
  \end{equation*}

  \vskip .5cm
  For random matrices\: $\scalemath{1.3}{W \in M_{\mathbb{C}}(N,N)}$

  \vskip .5cm
  Define chiral condensate and baryon number density
  %
  \begin{align*}
    \scalemath{1.1}{\langle \bar{\eta} \eta \rangle}\: &\scalemath{1.1}{= \partial_M \log \mathcal{Z}}\\
    \scalemath{1.1}{\langle \eta^{\dagger} \eta \rangle}\: &\scalemath{1.1}{= \partial_{\mu} \log \mathcal{Z}}
  \end{align*}
  
\end{frame}

\begin{frame}
  \frametitle{The Stephanov model}
  \framesubtitle{Properties}

  Solvable via bosonisation \inlinecite{Stephanov, 1996; Halasz, Jackson and
    Verbaarschot, 1997}
  %
  \[
    \mathcal{Z} = \int \mathrm{d} \sigma \mathrm{d} \sigma^* \, e^{- N\sigma^2}
    (\sigma^*\sigma + m(\sigma + \sigma^*) + m^2 - \mu^2)^N
  \]

  \vskip .35cm
  where $\scalemath{1.3}{\sigma \in M_{\mathbb{C}}(N_f,N_f)}$, \;for\;
  $\scalemath{1.3}{N_f = 1}$
  %
  \[
    \mathcal{Z} = \pi e^{-N m^2}\int_0^{\infty} \mathrm{d} u \, (u-\mu^2)^N
      I_0(2m N\sqrt{u})e^{-Nu}
  \]
\end{frame}

\begin{frame}
  \frametitle{The Stephanov model}
  \framesubtitle{Properties}

  The model has a phase transition as it develops non-zero baryon number
  density

  \vskip .75cm
  Solvable in the thermodynamic limit via a saddle point approximation

  \vskip .75cm
  In the chiral limit one finds $\scalemath{1.1}{\mu_c = 0.572...}$ and $\scalemath{1.1}{\mu_c = i}$

\end{frame}
