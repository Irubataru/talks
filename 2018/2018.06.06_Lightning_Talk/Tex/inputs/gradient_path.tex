\usetikzlibrary{fadings,decorations.pathmorphing}
% Code courtesy of https://tex.stackexchange.com/a/137438/39313

\makeatletter
\newif\iftikz@shading@path

\tikzset{
    % There are three circumstances in which the fading sep is needed:
    % 1. Arrows which do not update the bounding box (which is most of them).
    % 2. Line caps/joins and mitres that extend outside the natural bounding 
    %    box of the path (these are not calculated by PGF).
    % 3. Other reasons that haven't been anticipated.
    shading xsep/.store in=\tikz@pathshadingxsep,
    shading ysep/.store in=\tikz@pathshadingysep,
    shading sep/.style={shading xsep=#1, shading ysep=#1},
    shading sep=0.0cm,
}

\def\tikz@shadepath#1{% 
    % \tikz@addmode installs the `modes' (e.g., fill, draw, shade) 
    % to be applied to the path. It isn't usualy for doing more
    % changes to the path's construction.
    \iftikz@shading@path%
    \else%
        \tikz@shading@pathtrue%
        % Get the current path.
        \pgfgetpath\tikz@currentshadingpath%
        % Get the shading sep without setting any other keys.
        \begingroup%
            \pgfsys@beginscope% <- may not be necessary
            \tikzset{#1}%
            \xdef\tikz@tmp{\noexpand\def\noexpand\tikz@pathshadingxsep{\tikz@pathshadingxsep}%
                \noexpand\def\noexpand\tikz@pathshadingysep{\tikz@pathshadingysep}}%
            \pgfsys@endscope%
        \endgroup
        \tikz@tmp%
        % Get the boudning box of the current path size including the shading sep
        \pgfextract@process\pgf@shadingpath@southwest{\pgfpointadd{\pgfqpoint{\pgf@pathminx}{\pgf@pathminy}}%
            {\pgfpoint{-\tikz@pathshadingxsep}{-\tikz@pathshadingysep}}}%%
        \pgfextract@process\pgf@shadingpath@northeast{\pgfpointadd{\pgfqpoint{\pgf@pathmaxx}{\pgf@pathmaxy}}%
            {\pgfpoint{\tikz@pathshadingxsep}{\tikz@pathshadingysep}}}%
        % Clear the path
        \pgfsetpath\pgfutil@empty%                          
        % Save the current drawing mode and options.
        \let\tikz@options@saved=\tikz@options%
        \let\tikz@mode@saved=\tikz@mode%
        \let\tikz@options=\pgfutil@empty%
        \let\tikz@mode=\pgfutil@empty%
        % \tikz@options are processed later on.
        \tikz@addoption{%
            \pgfinterruptpath%
            \pgfinterruptpicture%
                \begin{tikzfadingfrompicture}[name=.]
                \pgfscope%
                    \tikzset{shade path/.style=}% Make absolutely sure shade path is not inherited.
                    \path \pgfextra{%
                        % Set the softpath. Any transformations,draw=none} in #1 will have no effect.
                        % This will *not* update the bounding box...
                        \pgfsetpath\tikz@currentshadingpath%
                        % ...so it is done manually.
                        \pgf@shadingpath@southwest
                        \expandafter\pgf@protocolsizes{\the\pgf@x}{\the\pgf@y}%
                        \pgf@shadingpath@northeast%
                        \expandafter\pgf@protocolsizes{\the\pgf@x}{\the\pgf@y}%
                        % Install the drawing modes and options.
                        \let\tikz@options=\tikz@options@saved%
                        \let\tikz@mode=\tikz@mode@saved%
                    };
                    % Now get the bounding box of the picture.
                    \xdef\pgf@shadingboundingbox@southwest{\noexpand\pgfqpoint{\the\pgf@picminx}{\the\pgf@picminy}}%
                    \xdef\pgf@shadingboundingbox@northeast{\noexpand\pgfqpoint{\the\pgf@picmaxx}{\the\pgf@picmaxy}}%
                    \endpgfscope
                \end{tikzfadingfrompicture}%
            \endpgfinterruptpicture%
            \endpgfinterruptpath%
            % Install a rectangle that covers the shaded/faded path picture.
            \pgftransformreset%
            \pgfpathrectanglecorners{\pgf@shadingboundingbox@southwest}{\pgf@shadingboundingbox@northeast}%
            %
            % Reset all modes.
            \let\tikz@path@picture=\pgfutil@empty%
            \tikz@mode@fillfalse%
            \tikz@mode@drawfalse%
            \tikz@mode@doublefalse%
            \tikz@mode@clipfalse%
            \tikz@mode@boundaryfalse%
            \tikz@mode@fade@pathfalse%
            \tikz@mode@fade@scopefalse%
            % Now install shading options.
            \tikzset{#1}%
            \tikz@mode%
            % Make the fading happen.
            \def\tikz@path@fading{.}%
            \tikz@mode@fade@pathtrue%
            \tikz@fade@adjustfalse%
            % Shift the fading to the mid point of the rectangle
            \pgfpointscale{0.5}{\pgfpointadd{\pgf@shadingboundingbox@southwest}{\pgf@shadingboundingbox@northeast}}%
            \edef\tikz@fade@transform{shift={(\the\pgf@x,\the\pgf@y)}}%
            \pgfsetfading{\tikz@path@fading}{\tikz@do@fade@transform}%
            \tikz@mode@fade@pathfalse%              
        }%
    \fi%
}
\tikzset{
    shade path/.code={%
        \tikz@addmode{\tikz@shadepath{#1}}%
    }
}

\makeatother
