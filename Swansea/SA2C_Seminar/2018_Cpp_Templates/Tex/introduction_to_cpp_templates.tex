\documentclass[14pt]{beamer}
\usepackage[T1]{fontenc}
\usepackage{url}
\usepackage{xcolor}
\usepackage{amsmath,varwidth}
\usepackage{tikz}
\usepackage{listings,algpseudocode}
\usepackage[quiet]{mathspec}

\setmainfont[
ItalicFont={Yanone Kaffeesatz Light},
Scale=1.3,
LetterSpace=2.0
]{Yanone Kaffeesatz Bold}

\setmonofont{Fira Code}

% --------------------- %
% Beamer theme settings %
% --------------------- %

\usetheme{CodeCourse}
\setbeamertemplate{navigation symbols}{} %Remove beamer navigation symbols

% ------------- %
% tikz settings %
% ------------- %

\usetikzlibrary{arrows.meta}

% ------------- %
% Font settings %
% ------------- %

% Load the eulervm package in the preamble to make sure one imports all the
% necessary mathematics symbols needed, then to get the font specification I
% want I need to lie to beamer about which font has which feature.

% Tells beamer that math mode should use serif fonts
% \usefonttheme[onlymath]{serif}

% Sets the serif font (which is the main font)
% \setmainfont[
%   ItalicFont={Linux Libertine O}
% ]{Euler Math}

% Sets the font beamer uses, which is sans-serif by default
% \setsansfont[
%   ItalicFont={Yanone Kaffeesatz Light},
%   Scale=1.3,
%   LetterSpace=2.0
% ]{Yanone Kaffeesatz Bold}

% Replaces every explicit \mathrm call to use this font
% \setmathrm{Linux Libertine O}

% -------------------------- %
% Standard LaTeX definitions %
% -------------------------- %

\DeclareGraphicsExtensions{.pdf}
\renewcommand\textbullet{\ensuremath{\bullet}}

\DeclareMathOperator{\tr}{tr}
\DeclareMathOperator{\real}{Re}

% --------------------------- %
% Useful function definitions %
% --------------------------- %

\newcommand{\tikzmark}[1]{\tikz[overlay,remember picture] \coordinate (#1);}

\newcommand\scalemath[2]{\scalebox{#1}{\mbox{\ensuremath{\displaystyle #2}}}}
\newcommand{\qmat}{\ensuremath{Q_{\mathrm{uark}}}}
\newcommand{\minus}{\scalebox{0.75}[1.0]{$-$}}
\newcommand{\inlinecite}[1]{\scalebox{0.6}{[#1]}}

\newcommand{\dimtext}[2]%
{
  \begin{tikzpicture}[overlay, remember picture]
    \fill[white, fill opacity=0.7] ( #1 -| current page.north west) -- ++(0,.8em) -- ++(\paperwidth,0) -- (#2 -| current page.north east)
   -- ++(0,-.5em) -- ++(-\paperwidth,0) -- cycle;
  \end{tikzpicture}
}


\usefonttheme{serif}

%\setbeameroption{show notes}

% Turn off page numbering
\setbeamertemplate{footline}{}

\title{Introduction to C++ templates}
\author{\texorpdfstring{%
    Jonas Rylund Glesaaen\newline%
    \fontsize{12pt}{12pt}\selectfont\texttt{jonas@glesaaen.com}%
  }{%
    Jonas Rylund Glesaaen}}

\date{\texorpdfstring{%
    SA{\color{SAACOrange}\textsuperscript{2}}C Seminar Series\newline{}February 9th 2018%
  }{%
    February 9th 2018}}

\begin{document}

\nocite{*}

\frame{\titlepage}

\section{Introduction introduction}
\frame{\sectionpage}

\begin{frame}[fragile]
  \frametitle{What is a template}

  \begin{minipage}{0.8\textwidth}
    Exactly what it says on the box.

    \vspace{1cm}
    A template is a template for constructing the thing it templates
  \end{minipage}

\end{frame}

\begin{frame}[fragile]
  \frametitle{What is a template}

  %\fontspec{Yanone Kaffeesatz Regular}
  \begin{minipage}{0.8\textwidth}
    Say you asked a colleague to implement a function that checks whether an
    integer is even or not
  
    \vspace{1.5cm}
  
    and then you get this...
  \end{minipage}

\end{frame}

\begin{frame}[fragile]
  \frametitle{What is a template}

  \begin{lstlisting}[escapeinside={(*}{*)}]
    bool is_even(unsigned i)
    {
      if (i == 0)
        return true;
      else if (i == 1)
        return false;
      else if (i == 2)
        return true;
      else if (i == 3)
        return false;
      /* ... */
      else if (i == 18446744073709551614)
        return true;
      else
        return false;
    }
  \end{lstlisting}

\end{frame}

\begin{frame}[fragile]
  \frametitle{What is a template}

  Similar to how control structures and functions help us not repeat ourselves
  in code, templates does this on the type level

  \vspace{.75cm}

  You can apply all the same techniques \\
  {\itshape\changefontsize{10pt}(more on this in the next talk)}

\end{frame}

\begin{frame}[fragile]
  \frametitle{What is a template}

  \begin{lstlisting}[escapeinside={(*}{*)}]
    double add(double a, double b)
    {
      return a + b;
    }

    double add(double a, int b)
    {
      return a + b;
    }

    double add(int a, double b)
    {
      return a + b;
    }

    int add(int a, int b)
    {
      return a + b;
    }
  \end{lstlisting}

\end{frame}

\begin{frame}[fragile]
  \frametitle{What is a template}

  \begin{lstlisting}[basicstyle=\codefontsize{12pt}]
template <typename T, typename U>
auto add(T a, U b)
{
  return a + b;
}
  \end{lstlisting}

  \vspace{.5cm}

  Usable by any type {\color{sblue}T} and {\color{sblue}U} pair that has a valid \\
  {\color{sblue}operator+}

\end{frame}

\section{Types of templates}
\frame{\sectionpage}

\begin{frame}[fragile]
  \frametitle{Types of templates : functions}

  Function templates are not functions

  \vspace{.5cm}

  \begin{lstlisting}[basicstyle=\codefontsize{12pt}]
template <typename Type>
bool not_empty(Type const& x)
{
  return ! x.empty();
}
  \end{lstlisting}

  \vspace{.5cm}

  If a specific instance isn't used it is not created

  \vspace{1mm}

  {\itshape\changefontsize{10pt}%
    (can you identify all the requirements on {\color{sblue}Type}?)}

\end{frame}

\begin{frame}[fragile]
  \frametitle{Types of templates : classes}

  Class templates are not classes

  \vspace{.4cm}

  \begin{lstlisting}[basicstyle=\codefontsize{10pt}]
template <typename Type>
class Linked_List
{
public:
  class Node
  {
  private:
    Type val;
    std::unique_ptr<Node> next;
  };

private:
  std::unique_ptr<Node> first_node;
};
  \end{lstlisting}

\end{frame}

\begin{frame}[fragile]
  \frametitle{Types of templates : aliases}

  \begin{onlyenv}<1>
      First thought when you see the {\color{sblue}using} keyword
    
      \vspace{.5cm}
    
      \begin{lstlisting}[basicstyle=\codefontsize{12pt}]
typedef(*\tikzmark{typedef}*) std::vector<double> Vector;
      \end{lstlisting}
    
      \vspace{.75cm}
    
      \begin{lstlisting}[basicstyle=\codefontsize{12pt}]
using(*\tikzmark{using}*) Vector = std::vector<double>;
      \end{lstlisting}
    
      \nointerlineskip
      \begin{tikzpicture}[overlay,remember picture, every node/.style={scale=0.8}, every path/.style={line width=1pt}]
        \draw[<-,>=stealth] ([shift={(-.5cm,-5pt)}] using) %
          .. controls +(0,-.35cm) and +(-.65cm,0) .. %
          +(.65cm,-.35cm) node [right] {C++11 Syntax};
        \draw[<-,>=stealth] ([shift={(-.75cm,-5pt)}] typedef) %
          .. controls +(0,-.35cm) and +(-.5cm,0) .. %
          +(.5cm,-.35cm) node [right] {C++03 Syntax};
      \end{tikzpicture}
  \end{onlyenv}

  \begin{onlyenv}<2>

    Also introduces alias templates \\
    {\changefontsize{10pt}...they are actually aliases}

    \vspace{.25cm}

    \begin{lstlisting}[basicstyle=\codefontsize{11pt}]
template <Typename T>
using String_Map = std::map<std::string, T>;
    \end{lstlisting}

    \vfill

    These are the exact same types

    \begin{lstlisting}[basicstyle=\codefontsize{11pt},morekeywords={String_Map}]
std::is_same<
  String_Map<double>,
  std::map<std::string, double>
>::value(*\tikzmark{is-same-value}*)
    \end{lstlisting}
    
    \nointerlineskip
    \begin{tikzpicture}[overlay,remember picture]
      \draw[pointy arrow] ([shift={(1mm,.12em)}] is-same-value) %
        --
        +(7cm,0) node [right, scale=0.9] {{\color{sgreen}true}};
    \end{tikzpicture}

  \end{onlyenv}

\end{frame}

\begin{frame}[fragile]
  \frametitle{Types of templates : variables}

  \begin{onlyenv}<1>
    Template variables are not variables

    \vspace{.5cm}

    \begin{lstlisting}[basicstyle=\codefontsize{12pt}]
template <typename T>
constexpr bool is_double = 
      std::is_same<T, double>::value;
    \end{lstlisting}

    \vspace{.5cm}

    \begin{lstlisting}[basicstyle=\codefontsize{12pt}]
is_double<int>;(*\tikzmark{is-double-int}*)
is_double<double>;(*\tikzmark{is-double-double}*)
    \end{lstlisting}

    \nointerlineskip
    \begin{tikzpicture}[overlay,remember picture]
      \draw[pointy arrow,thin] ([shift={(1mm,.25ex)}] is-double-int) -- +(5cm,0)
        coordinate (layer1)
        node[right,scale=0.8] {{\color{sorange}false}};
      \draw[transform canvas={yshift=.25ex},pointy arrow,thin] ([xshift=1mm] is-double-double) -- (is-double-double -| layer1)
        node[right,scale=0.8] {{\color{sgreen}true}};
    \end{tikzpicture}

    \CPPFourteen
  \end{onlyenv}

  \begin{onlyenv}<2>
    We used to do this

    \vspace{.5cm}

    \begin{lstlisting}[basicstyle=\codefontsize{12pt}]
template <typename T>
struct is_double
{
  static constexpr bool value = 
        std::is_same<T, double>::value
};
    \end{lstlisting}

    \vspace{.5cm}

    \begin{lstlisting}[basicstyle=\codefontsize{12pt}]
is_double<int>::value;(*\tikzmark{is-double-int}*)
is_double<double>::value;(*\tikzmark{is-double-double}*)
    \end{lstlisting}

    \nointerlineskip
    \begin{tikzpicture}[overlay,remember picture]
      \draw[pointy arrow,thin] ([shift={(1mm,.25ex)}] is-double-int) -- +(3.25cm,0)
        coordinate (layer1)
        node[right,scale=0.8] {{\color{sorange}false}};
      \draw[transform canvas={yshift=.25ex},pointy arrow,thin] ([xshift=1mm] is-double-double) -- (is-double-double -| layer1)
        node[right,scale=0.8] {{\color{sgreen}true}};
    \end{tikzpicture}
    
  \end{onlyenv}

\end{frame}

\begin{frame}

  \begin{center}\fontsize{16pt}{16pt}\selectfont\color{Tropiteal}
    Fundamental theorem of software engineering
  \end{center}

  \vspace{.25cm}
  \begin{center}
    \begin{minipage}{8cm}
      We can solve any problem by introducing an extra level of indirection.
    \end{minipage}
  \end{center}

  \vspace{.25cm}
  \begin{flushright}\color{ICantExpress!50!WhiteTrash}
    David J. Wheeler
  \end{flushright}

\end{frame}

\begin{frame}

  \begin{center}\fontsize{16pt}{16pt}\selectfont\color{Tropiteal}
    Fundamental theorem of templates
  \end{center}

  \vspace{.25cm}
  \begin{center}
    \begin{minipage}{8cm}
      Any finite problem can be solved with a finite number of structs.
    \end{minipage}
  \end{center}

  \vspace{.25cm}
  \begin{flushright}\color{ICantExpress!50!WhiteTrash}
    Jonas R. Glesaaen
  \end{flushright}

\end{frame}

\section{Template instantation}
\frame{\sectionpage}

\begin{frame}[fragile]
  \frametitle{Template instantiation}

  Templates are instantiated when they are used\\
  {\itshape\changefontsize{10pt}(which is why linking can be counterintuitive)}

  \vspace{.75cm}

  Can also instantiate them explicitly

  \vspace{.25cm}

  \begin{lstlisting}[basicstyle=\codefontsize{10pt}]
template bool is_empty(std::vector<double>);
template class Linked_List<int>;
template bool is_double<char>;
  \end{lstlisting}

\end{frame}

\begin{frame}[fragile]
  \frametitle{Template instantiation}

  C++ will never instansiate anything it \\ doesn't 100\% need
  {\itshape\changefontsize{10pt}(more or less true)}

  \vspace{.25cm}

  \begin{lstlisting}[basicstyle=\codefontsize{12pt}]
template <typename T>
struct invalid
{
  static_assert(sizeof(T) == 0);
};
  \end{lstlisting}

  \begin{overlayarea}{\textwidth}{1cm}

    \begin{onlyenv}<1>
      \begin{lstlisting}[basicstyle=\codefontsize{12pt}]
invalid<int> *inv_ptr;(*\tikzmark{inv-ptr}*)
      \end{lstlisting}

      \nointerlineskip
      \begin{tikzpicture}[overlay,remember picture]
        \draw[pointy arrow,thin] ([shift={(1mm,.25ex)}] inv-ptr) -- +(1.5cm,0)
          node[right,scale=0.75] {\color{Tropiteal}Compiles};
      \end{tikzpicture}
    \end{onlyenv}

    \begin{onlyenv}<2>
      \begin{lstlisting}[basicstyle=\codefontsize{12pt}]
invalid<int> inv_val;(*\tikzmark{inv-val}*)
      \end{lstlisting}

      \nointerlineskip
      \begin{tikzpicture}[overlay,remember picture]
        \draw[pointy arrow,thin] ([shift={(1mm,.25ex)}] inv-val) -- +(1.5cm,0)
          node[right,scale=0.75] {\color{Marty}Compile error};
      \end{tikzpicture}
    \end{onlyenv}
    
  \end{overlayarea}

\end{frame}

\begin{frame}[fragile]
  \frametitle{Template instantiation}

    Does it compile?

    \vspace{.25cm}

  \begin{lstlisting}[basicstyle=\codefontsize{12pt}]
template <typename T>
struct invalid
{
  void foo()
  {
    static_assert(sizeof(T) == 0);
  }
};

invalid<int> inv_var;(*\tikzmark{does-it-compile}*)
  \end{lstlisting}

  \begin{onlyenv}<2>
    \nointerlineskip
    \begin{tikzpicture}[overlay,remember picture]
      \node[right] at ([shift={(1.5,.25ex)}] does-it-compile)
         {\color{Tropiteal}Yes, it does};
    \end{tikzpicture}
  \end{onlyenv}

\end{frame}

\begin{frame}[fragile]
  \frametitle{Template instantiation}

    Does it compile?

    \vspace{.25cm}

  \begin{lstlisting}[basicstyle=\codefontsize{12pt}]
template <typename T>
struct invalid
{
  void foo()
  {
    static_assert(sizeof((*\tikzmark{char-begin}*)char) == 0);
  }
};

invalid<int> inv_var;(*\tikzmark{does-it-compile}*)
  \end{lstlisting}

  \begin{onlyenv}<1>
    \nointerlineskip
    \begin{tikzpicture}[overlay,remember picture]
      \draw[FeebleWeek,thick] ([yshift=-3pt] char-begin) -- +(1.05cm,0);
    \end{tikzpicture}
  \end{onlyenv}

  \begin{onlyenv}<2>
    \nointerlineskip
    \begin{tikzpicture}[overlay,remember picture]
      \draw[Marty,thick] ([yshift=-3pt] char-begin) -- +(1.05cm,0);
      \node[right] at ([shift={(1.5,.25ex)}] does-it-compile)
         {\color{Marty}No, it does not};
    \end{tikzpicture}
  \end{onlyenv}

\end{frame}

\section{Dependent types}
\frame{\sectionpage}

\begin{frame}[fragile]
  \frametitle{Dependent types : typename}

  A dependent name is a name that depends on an unknown template type

  \vspace{.25cm}

  \begin{overlayarea}{\textwidth}{3cm}
    \begin{onlyenv}<1>
      \begin{lstlisting}[basicstyle=\codefontsize{12pt}]
template <typename T>
void foo(int x)
{
  auto v = T::U(x);(*\tikzmark{dependent-1}*)
}
      \end{lstlisting}
    \end{onlyenv}

    \begin{onlyenv}<2>
      \begin{lstlisting}[basicstyle=\codefontsize{12pt}]
template <typename T>
void foo(int x)
{
  auto v = typename T::U(x);(*\tikzmark{dependent-2}*)
}
      \end{lstlisting}
    \end{onlyenv}
  \end{overlayarea}
  
  \nointerlineskip
  \begin{tikzpicture}[overlay,remember picture]
    \only<1>{
      \draw[pointy arrow,thin] ([shift={(1mm,.25ex)}] dependent-1) -- +(2cm,0)
        node[right,scale=0.75] (correct) {Is it a function call?};
      \draw[thin] ([shift={(5mm,.25ex)}] dependent-1) %
        .. controls +(1cm,0) and +(-1cm,0) %
        .. +(2cm-4mm,-1cm) node[right,scale=0.75] {Is it a variable declaration?};
      \draw[thin, transform canvas={yshift=2pt}] (correct.south west) -- ([xshift=-1pt]correct.south east);
    }
    \only<2>{
      \draw[pointy arrow,thin] ([shift={(1mm,.25ex)}] dependent-2) %
        .. controls +(1.5cm,0) and +(0,.75cm) %
        .. +(1.5cm,-.75cm) node[below,scale=0.75] {Ok, it is a declaration};
      }
  \end{tikzpicture}

\end{frame}

\begin{frame}[fragile]
  \frametitle{Dependent types : template}

  Can also have dependent names that are templates

  \vspace{.25cm}

  \begin{overlayarea}{\textwidth}{3cm}
    \begin{onlyenv}<1>
      \begin{lstlisting}[basicstyle=\codefontsize{12pt}]
template <typename T, int N>
void foo(int x)
{
  auto v = T::U<N>(x);(*\tikzmark{dependent-1}*)
}
      \end{lstlisting}
    \end{onlyenv}

    \begin{onlyenv}<2>
      \begin{lstlisting}[basicstyle=\codefontsize{12pt}]
template <typename T, int N>
void foo(int x)
{
  auto v = T::template U<N>(x);(*\tikzmark{dependent-2}*)
}
      \end{lstlisting}
    \end{onlyenv}

    \begin{onlyenv}<3>
      \begin{lstlisting}[basicstyle=\codefontsize{12pt}]
template <typename T, int N>
void foo(int x)
{
  auto v = typename T::template U<N>(x);(*\tikzmark{dependent-3}*)
}
      \end{lstlisting}
    \end{onlyenv}

  \end{overlayarea}
  
  \nointerlineskip
  \begin{tikzpicture}[overlay,remember picture]
    \only<1>{
      \draw[pointy arrow,thin] ([shift={(1mm,.25ex)}] dependent-1) -- +(2cm,0)
        node[right,scale=0.75] (correct) {Value comparison?};
      \draw[thin] ([shift={(3mm,.25ex)}] dependent-1) %
        .. controls +(1cm,0) and +(-1cm,0) %
        .. +(2cm-2mm,-1cm) node[right,scale=0.75] {Function call?};
      \draw[thin] ([shift={(3mm,.25ex)}] dependent-1) %
        .. controls +(1cm,0) and +(-1cm,0) %
        .. +(2cm-2mm,-2cm) node[right,scale=0.75] {Constructor call?};

      \draw[thin, transform canvas={yshift=2pt}] (correct.south west) -- ([xshift=-1pt]correct.south east);
    }
    \only<2>{
      \draw[pointy arrow,thin] ([shift={(-.6cm,-5pt)}] dependent-2) -- +(0,-1cm)
        node[below,scale=0.75] (correct) {Ok, function call};
    }
    \only<3>{
      \draw[pointy arrow,thin] ([shift={(-.6cm,-5pt)}] dependent-3) %
        .. controls +(0,-.75cm) and +(0,.75cm) %
        .. +(-2cm+.5mm,-1cm) node[below,scale=0.75] (correct) {Ok, type constructor};
    }

  \end{tikzpicture}
\end{frame}

\section{Template deduction}
\frame{\sectionpage}

\begin{frame}[fragile]
  \frametitle{Template deduction : by value}

  Most straight forward is to explicitly specify the template types

  \begin{overlayarea}{\textwidth}{4cm}
    \begin{onlyenv}<1-2>
    \begin{lstlisting}[basicstyle=\codefontsize{12pt}]
template <typename T, typename U>
auto(*\tmark{return-type}*) max(T a, U b);

max<int(*\tmark{type-1}*),int(*\tmark{type-2}*)>(1(*\tmark{arg-1}*), 5.(*\tmark{arg-2}*));
    \end{lstlisting}
    \end{onlyenv}

    \nointerlineskip
    \begin{tikzpicture}[overlay,remember picture]
      \only<1>{
        \draw[pointy arrow, thin] ([shift={(-4mm,-2pt)}] type-1) %
          -- +(0,-1cm) node[below, scale=0.8] {{\color{sblue}T} = {\color{sgreen}int}};
        \draw[pointy arrow, thin] ([shift={(-4mm,-2pt)}] type-2) %
          .. controls +(0,-.75cm) and +(-.25cm,0) %
          .. +(.25cm,-.75cm) node[right, scale=0.8] {{\color{sblue}U} = {\color{sgreen}int}};
      }
      \only<2>{
        \coordinate (mid-arg) at ([xshift=-1.5mm] $(arg-1)!0.5!(arg-2)$);
        \draw[pointy arrow, thin] ([shift={(-1mm,-2pt)}] arg-1) %
          .. controls +(0,-.5cm) and +(0,.5cm) %
          .. ([yshift=-1cm] mid-arg);
        \draw[pointy arrow, thin] ([shift={(-2.5mm,-2pt)}] arg-2) %
          .. controls +(0,-.5cm) and +(0,.5cm) %
          .. ([yshift=-1cm] mid-arg) %
          node[below, scale=0.8] {Irrelevant for type deduction};
        \draw[pointy arrow, thin] ([shift={(-2.5mm,-2pt)}] return-type) %
          .. controls +(.1cm,-.5cm) and +(-5cm,0) %
          .. +(6cm,-.25cm) %
          node[right,align=center, scale=0.8,yshift=-6.5mm] %
            {Return type\\doesn't contribute\\to type deduction};
      }
    \end{tikzpicture}

    \only<3>{%
      \vspace{.5cm}

      If a template is explicitly declared it does not enter into argument type
      deduction
    }
    \end{overlayarea}

\end{frame}

\begin{frame}[fragile]
  \frametitle{Template deduction : by value}

  Every type that is not explicitly defined must be deduced from the function
  arguments

  \hspace{.25cm}

  Deduction flow

  \begin{enumerate} \changefontsize{12pt}\itshape
    \item Explicit types are set
    \item Each argument factors into type dediction separately
    \begin{itemize} \changefontsize{10pt}\itshape
      \item This happens "in parallel"
      \item Resulting deduction is either full or partial
    \end{itemize}
    \item The deduced types are checked for consistency
    \item Template checks that all types are fully deduced
  \end{enumerate}

\end{frame}

\begin{frame}[fragile]
  \frametitle{Template deduction : by value}

  \begin{overlayarea}{\textwidth}{3cm}
    \begin{onlyenv}<1-2>
      \begin{lstlisting}[basicstyle=\codefontsize{12pt}]
template <typename T>
T max(T a, T b);

max(4(*\tmark{arg-1}*), 5.(*\tmark{arg-2}*));
      \end{lstlisting}
    \end{onlyenv}
    \begin{onlyenv}<3>
      \begin{lstlisting}[basicstyle=\codefontsize{12pt}]
template <typename T>
T max(T a, T b);

max<double(*\tikzmark{type-arg}*)>(4(*\tmark{arg-3}*), 5.(*\tmark{arg-4}*));
        \end{lstlisting}
    \end{onlyenv}
  \end{overlayarea}

  \vspace{2cm}

  \nointerlineskip
  \begin{tikzpicture}[overlay,remember picture]
    \only<2>{
      \draw[pointy arrow, thin] ([shift={(-1mm,-2pt)}] arg-1) %
        .. controls +(0,-2cm) and +(-2cm,0) %
        .. +(2cm, -2cm) %
        node[right, scale=0.8] (node-1) {{\color{sblue}T} = {\color{sgreen}int}};
      \draw[pointy arrow, thin] ([shift={(-2.5mm,-2pt)}] arg-2) %
        .. controls +(0,-1cm) and +(-1cm,0) %
        .. ([yshift=1cm] node-1.west) %
        node[right, scale=0.8] (node-2) {{\color{sblue}T} = {\color{sgreen}double}};

      \draw[%
        thick,
        decorate,
        decoration={brace,amplitude=5pt}
      ] (node-2.north east) -- (node-1.south east -| node-2.east) %
        node[midway, right, scale=0.8, xshift=2mm, font=\color{Marty}] {Incompatible};
    }

      \draw[pointy arrow, thin] ([shift={(-4mm,-2pt)}] type-1) %
        -- +(0,-1cm) node[below, scale=0.8] {{\color{sblue}T} = {\color{sgreen}int}};
      \draw[pointy arrow, thin] ([shift={(-4mm,-2pt)}] type-2) %
        .. controls +(0,-.75cm) and +(-.25cm,0) %
        .. +(.25cm,-.75cm) node[right, scale=0.8] {{\color{sblue}U} = {\color{sgreen}int}};

    \only<3>{
      \draw[pointy arrow, thin] ([shift={(-7.5mm,-2pt)}] type-arg) %
        -- +(0,-2cm) node[below, scale=0.8] {{\color{sblue}T} = {\color{sgreen}double}};
      \draw[pointy arrow, thin] ([shift={(-1mm,-2pt)}] arg-3) %
        .. controls +(0,-1cm) and +(-2cm,0) %
        .. +(2cm, -1cm) %
        node[right, scale=0.8] (irrelevant) {Irrelevant};
      \draw[pointy arrow, thin] ([shift={(-2.5mm,-2pt)}] arg-4) %
        .. controls +(0,-1cm) and +(-1cm,0) %
        .. (irrelevant.west);
    }
  \end{tikzpicture}

\end{frame}

\begin{frame}[fragile]
  \frametitle{Template deduction : by value}

  Puzzle \#1

  \begin{lstlisting}[basicstyle=\codefontsize{10pt}]
template <typename T, bool B>
struct My_Type{};

template <typename T, typename U>
void foo(
  My_Type<T, !std::is_same_v<T, U>>,
  My_Type<U, sizeof(U) == sizeof(T)>);
  \end{lstlisting}

  \begin{onlyenv}<1>
    \begin{lstlisting}[basicstyle=\codefontsize{10pt}]
int main()
{
  foo(My_Type<int,true>{}, 
      My_Type<double,true>{});
}
      
    \end{lstlisting}
  \end{onlyenv}

  \begin{onlyenv}<2>
    \begin{lstlisting}[basicstyle=\codefontsize{10pt}]
int main()
{
  foo(My_Type<int,true>{}, 
      My_Type<unsigned,true>{});
}
      
    \end{lstlisting}
  \end{onlyenv}

\end{frame}

\begin{frame}[fragile]
  \frametitle{Template deduction : by value}

    Puzzle \#2

  \begin{lstlisting}[basicstyle=\codefontsize{10pt}]
template <typename T, typename U>
void foo(
  std::array<T, sizeof(U)>,
  std::array<U, sizeof(T)>);
  \end{lstlisting}

  \begin{onlyenv}<1>
  \begin{lstlisting}[basicstyle=\codefontsize{10pt}]
int main()
{
  foo(std::array<int, 4>,
      std::array<double, 4>);
}
    
  \end{lstlisting}
  \end{onlyenv}

  \begin{onlyenv}<2>
    \begin{lstlisting}[basicstyle=\codefontsize{10pt}]
int main()
{
  foo(std::array<int, 8>,
      std::array<double, 4>);
}
      
    \end{lstlisting}
  \end{onlyenv}

\end{frame}

\begin{frame}[fragile]
  \frametitle{Template deduction : references}

  \begin{columns}
    \column{.6\textwidth}
    {\changefontsize{16pt}{\color{Tropiteal}lvalues} (\&)}\\
    \hspace{5mm}anything with a tag\\
    \hspace{5mm}anything that will be missed

    \vspace{.5cm}

    {\changefontsize{16pt}{\color{Tropiteal}rvalues} (\&\&)\tmark{rval-ref}}\\
    \hspace{5mm}intermediate values\\
    \hspace{5mm}things no one will miss

    \column{.4\textwidth}

    \begin{tikzpicture}[remember picture]
      \node[draw=Tropiteal] (col-2) {%
        \begin{minipage}{0.86\textwidth}
          Reference collapse

          \vspace{.5cm}

          \& + \& = \&\\
          \&\& + \& = \&\\
          \& + \&\& = \&\\
          \&\& + \&\& = \&\&
        \end{minipage}};
    \end{tikzpicture}
  \end{columns}

  \vspace{1cm}

  \nointerlineskip
  \begin{tikzpicture}[overlay,remember picture]
    \draw[pointy arrow, rounded corners=10pt] %
    ([shift={(1mm, 2mm)}] rval-ref) %
    -| +(3cm, -2cm) %
    node[below, scale=0.8,align=center] (ref-desc) {Also called\\universal refrences};

    \draw[pointy arrow,rounded corners] ([yshift=-1mm] col-2.south) |- (ref-desc.east);
  \end{tikzpicture}

\end{frame}

\begin{frame}[fragile]
  \frametitle{Template deduction : references}

  Template deduction uses reference collapse so that the function signature
  matches the call

  \begin{overlayarea}{\textwidth}{5cm}
    \begin{onlyenv}<1>
      \begin{lstlisting}[basicstyle=\codefontsize{12pt}]
template <typename T>
auto foo(T &);
      \end{lstlisting}
    \end{onlyenv}

    \begin{onlyenv}<2>
      \begin{lstlisting}[basicstyle=\codefontsize{12pt}]
template <typename T>
auto foo(T const &);
      \end{lstlisting}
    \end{onlyenv}

    \begin{onlyenv}<3>
      \begin{lstlisting}[basicstyle=\codefontsize{12pt}]
template <typename T>
auto foo(T &&);
      \end{lstlisting}
    \end{onlyenv}


    \begin{lstlisting}[basicstyle=\codefontsize{12pt}]
int i;

foo(i);(*\tmark{ex-1}*)
foo(std::move(i));(*\tmark{ex-2}*)
foo(5);(*\tmark{ex-3}*)
    \end{lstlisting}
  \end{overlayarea}

  \nointerlineskip
  \begin{tikzpicture}[overlay,remember picture]
    \only<1-2>{
      \draw[pointy arrow] ([shift={(1mm,1mm)}] ex-1) -- +(5cm,0) %
      node[right, scale=0.8] (node-1) {Ok! {\color{sblue}T} = {\color{sgreen}int}};
    }
    \only<1>{
      \draw[pointy arrow, transform canvas={yshift=1mm}] ([xshift=1mm] ex-2) -- (ex-2 -| node-1.west) %
      node[right, scale=0.8, font=\color{Marty}] {Compile error};
      \draw[pointy arrow, transform canvas={yshift=1mm}] ([xshift=1mm] ex-3) -- (ex-3 -| node-1.west) %
      node[right, scale=0.8, font=\color{Marty}] {Compile error};
    }
    \only<2>{
      \draw[pointy arrow, transform canvas={yshift=1mm}] ([xshift=1mm] ex-2) -- (ex-2 -| node-1.west) %
      node[right, scale=0.8] {Ok! {\color{sblue}T} = {\color{sgreen}int}};
      \draw[pointy arrow, transform canvas={yshift=1mm}] ([xshift=1mm] ex-3) -- (ex-3 -| node-1.west) %
      node[right, scale=0.8] {Ok! {\color{sblue}T} = {\color{sgreen}int}};
    }
    \only<3>{
      \draw[pointy arrow] ([shift={(1mm,1mm)}] ex-1) -- +(5cm,0) %
      node[right, scale=0.8] (node-1) {Ok! {\color{sblue}T} = {\color{sgreen}int}\&};
      \draw[pointy arrow, transform canvas={yshift=1mm}] ([xshift=1mm] ex-2) -- (ex-2 -| node-1.west) %
      node[right, scale=0.8] {Ok! {\color{sblue}T} = {\color{sgreen}int}};
      \draw[pointy arrow, transform canvas={yshift=1mm}] ([xshift=1mm] ex-3) -- (ex-3 -| node-1.west) %
      node[right, scale=0.8] {Ok! {\color{sblue}T} = {\color{sgreen}int}};
    }
  \end{tikzpicture}

\end{frame}

\begin{frame}[fragile]
  \frametitle{Template deduction : variadic}

  Variadic template packs can hold any number of types

  \vspace{.25cm}
  
  \begin{lstlisting}[basicstyle=\codefontsize{12pt}]
template <typename... Tp>
auto foo(Tp...);

foo();(*\tmark{ex-0}*)
foo(3, 1l, 'a');(*\tmark{ex-1}*)
foo<double>(3, 1l, 'a');(*\tmark{ex-2}*)
foo<int,int,int>(2,4);(*\tmark{ex-3}*)
  \end{lstlisting}

  \nointerlineskip
  \begin{tikzpicture}[overlay,remember picture]
    \draw[pointy arrow] ([shift={(1mm,1mm)}] ex-1) -- +(2.7cm,0) %
      node[right, scale=0.8] (node-1) {%
        {\color{sblue}Tp} = <{\color{sgreen}int}, {\color{sgreen}long}, {\color{sgreen}char}>};
    \draw[pointy arrow, transform canvas={yshift=1mm}] ([xshift=1mm] ex-2) -- (ex-2 -| node-1.west) %
      node[right, scale=0.8] {
        {\color{sblue}Tp} = <{\color{sgreen}double}, {\color{sgreen}long}, {\color{sgreen}char}>};
    \draw[pointy arrow, transform canvas={yshift=1mm}] ([xshift=1mm] ex-3) -- (ex-3 -| node-1.west) %
      node[right, scale=0.8, font=\color{Marty}] (node-3) {Compile error};
    \draw[pointy arrow, transform canvas={yshift=1mm}] ([xshift=1mm] ex-0) -- (ex-0 -| node-1.west) %
      node[right, scale=0.8] {
        {\color{sblue}Tp} = <>};

    \coordinate (node-3-west) at (node-3.west);
    \node[
      scale=0.6,
      font=\color{Marty},
      below of=node-3-west,
      anchor=west,
      transform canvas={yshift=2mm,xshift=.5mm}
    ] {Expected 3 arguments...};
  \end{tikzpicture}

  \vspace{.25cm}

\end{frame}

\begin{frame}[fragile]
  \frametitle{Template deduction : variadic}

  Template packs gobble up everything to the right

  \vspace{.5cm}
  
  \begin{lstlisting}[basicstyle=\codefontsize{12pt}]
template <typename T, typename... Tp>
auto foo(Tp..., T);

foo(3, 1l, 'a');(*\tmark{ex-1}*)
  \end{lstlisting}

  \vspace{1cm}

  \begin{lstlisting}[basicstyle=\codefontsize{12pt}]
foo<char, int, long>(
  3, 1l, 'a');(*\tmark{ex-2}*)
  \end{lstlisting}

  \nointerlineskip
  \begin{tikzpicture}[overlay,remember picture]
    \draw[pointy arrow] ([shift={(1mm,1mm)}] ex-1) -- +(2.7cm,0) %
      node[right, scale=0.8, align=left, yshift=-6mm] (node-1) {%
        {\color{Marty} Compile error}\\
        {\changefontsize{10}\color{Marty}Cannot determine T}\\
        {\changefontsize{10}{\color{sblue}Tp} = <{\color{sgreen}int}, {\color{sgreen}long}, {\color{sgreen}char}>}};
    \draw[pointy arrow, transform canvas={yshift=1mm}] ([xshift=1mm] ex-2) -- (ex-2 -| node-1.west) %
      node[right, scale=0.8, align=left,yshift=-3mm] {
        {\color{sblue}T} = {\color{sgreen}double}\\
        {\color{sblue}Tp} = <{\color{sgreen}int}, {\color{sgreen}long}>};
  \end{tikzpicture}

  \vspace{.25cm}

\end{frame}

\begin{frame}[fragile]
  \frametitle{Template deduction : classes}

  Introduced in C++17, can now write

  \vspace{.25cm}

  \begin{overlayarea}{\textwidth}{5cm}
  \begin{lstlisting}[basicstyle=\codefontsize{12pt}]
template <typename T>
class My_Class
{
public:
  My_Class(T val);
};
  \end{lstlisting}
  \begin{onlyenv}<2>
  \begin{lstlisting}[basicstyle=\codefontsize{12pt},morekeywords={My_Class}]
My_Class(double) -> My_Class<int>;(*\tmark{deduct-guide}*)
  \end{lstlisting}
  \end{onlyenv}
  \end{overlayarea}

  \begin{lstlisting}[basicstyle=\codefontsize{12pt},morekeywords={My_Class}]
My_Class inst {5.0};(*\tmark{inst-1}*)
  \end{lstlisting}

  \nointerlineskip
  \begin{tikzpicture}[overlay,remember picture]
    \only<1>{
      \draw[pointy arrow] ([shift={(1mm,1.5mm)}] inst-1) -- +(1cm,0) %
        node[right,scale=0.8] {{\color{sgreen}My\_Class}<{\color{sgreen}double}>};
    }
    \only<2>{
      \draw[pointy arrow] ([shift={(1mm,1.5mm)}] inst-1) -- +(1cm,0) %
        node[right,scale=0.8] {{\color{sgreen}My\_Class}<{\color{sgreen}int}>};

      \draw[pointy arrow] ([shift={(-2cm, 5mm)}] deduct-guide) -- +(0,1cm)
        node[above,scale=0.8,xshift=1cm] {Deduction guide};
    }
  \end{tikzpicture}

\end{frame}

\begin{frame}[fragile]
  \frametitle{Template deduction : classes}

  More or less syntactic sugar for this

  \vspace{.25cm}

  \begin{lstlisting}[basicstyle=\codefontsize{12pt}]
template <typename T>
class My_Class
{ ... };
  \end{lstlisting}

  \begin{lstlisting}[basicstyle=\codefontsize{12pt}, morekeywords={My_Class}]
template <typename T>
auto make_cls(T t)
{
  return My_Class<T>{t};
}

auto inst = make_cls(5.0);
  \end{lstlisting}

\end{frame}

\section{Template specialisation}
\frame{\sectionpage}

\begin{frame}[fragile]
  \frametitle{Template specialisation : full}

  Can specialise templates

  \begin{overlayarea}{\textwidth}{6.5cm}
  \begin{lstlisting}[basicstyle=\codefontsize{12pt}]
template <typename T>
T null()
{
  return T{0};
}
  \end{lstlisting}

  \begin{onlyenv}<1>
  \begin{lstlisting}[basicstyle=\codefontsize{12pt}]
template <>
std::string null<std::string>()
{
  return "";
}
  \end{lstlisting}
  \end{onlyenv}

  \begin{onlyenv}<2>
  \begin{lstlisting}[basicstyle=\codefontsize{12pt}]
template <>
std::string null<>()
{
  return "";
}
  \end{lstlisting}
  \end{onlyenv}

  \begin{onlyenv}<3>
  \begin{lstlisting}[basicstyle=\codefontsize{12pt}]
template <>
std::string null()
{
  return "";
}
  \end{lstlisting}
  \end{onlyenv}

  \begin{onlyenv}<4-5>
  \begin{lstlisting}[basicstyle=\codefontsize{12pt}]
template <>
std::string null() = delete;
  \end{lstlisting}

  \end{onlyenv}
  \only<5>{
    \vspace{.25cm}

    These are all examples of full specialisations\\
    {\itshape\changefontsize{10pt}(not all that interesting)}
  }

  \end{overlayarea}

\end{frame}

\begin{frame}[fragile]
  \frametitle{Template specialisation : partial}

  {\changefontsize{12pt}
  Partial template specialisations: pattern matching \textbackslash{}o/}

  \begin{overlayarea}{\textwidth}{5cm}
  \begin{lstlisting}[basicstyle=\codefontsize{10pt}]
template <typename T>
struct is_pointer_type(*\tikzmark{template-1}*)
{
  static constexpr bool value = false;
};
  \end{lstlisting}

  \begin{onlyenv}<1-5>
  \begin{lstlisting}[basicstyle=\codefontsize{10pt}]
template <typename T>
struct is_pointer_type<T*>(*\tikzmark{template-2}*)
{
  static constexpr bool value = true;
};
  \end{lstlisting}
  \end{onlyenv}

  \begin{onlyenv}<6>
  \begin{lstlisting}[basicstyle=\codefontsize{10pt}]
template <typename T>
struct is_pointer_type<std::shared_ptr<T>>(*\tikzmark{template-3}*)
{
  static constexpr bool value = true;
};
  \end{lstlisting}
  \end{onlyenv}
  \end{overlayarea}

  \begin{overlayarea}{\textwidth}{2cm}
    \begin{onlyenv}<1-2>
      \begin{lstlisting}[basicstyle=\codefontsize{10pt}]
is_pointer_type<int>::value;(*\tmark{ex-1}*)
      \end{lstlisting}
    \end{onlyenv}
    \begin{onlyenv}<3>
      \begin{lstlisting}[basicstyle=\codefontsize{10pt}]
is_pointer_type<int*>::value;(*\tmark{ex-2}*)
      \end{lstlisting}
    \end{onlyenv}
    \begin{onlyenv}<4>
      \begin{lstlisting}[basicstyle=\codefontsize{10pt}]
is_pointer_type<int**>::value;(*\tmark{ex-3}*)
      \end{lstlisting}
    \end{onlyenv}
    \begin{onlyenv}<5-6>
      \begin{lstlisting}[basicstyle=\codefontsize{10pt}]
is_pointer_type<
  std::shared_ptr<int>>::value;(*\tmark{ex-4}*)
      \end{lstlisting}
    \end{onlyenv}
  \end{overlayarea}

  \nointerlineskip
  \begin{tikzpicture}[overlay,remember picture]
    \coordinate (margin) at ([xshift=-2cm] current page.south east);
    \only<2>{
      \draw[pointy arrow,transform canvas={yshift=1mm}] ([xshift=1mm] ex-1) -- (ex-1 -| margin)
        node[right,scale=0.8,font=\color{sorange}] (value) {false};
      \draw[pointy arrow, rounded corners] ([shift={(1mm,1mm)}] template-1) -| (value);
      \path[transform canvas={yshift=1mm}] (template-1) -- (template-1 -| value)
        node[midway, above, scale=0.8] {{\color{sblue}T} = {\color{sgreen}int}};
    }
    \only<3>{
      \draw[pointy arrow,transform canvas={yshift=1mm}] ([xshift=1mm] ex-2) -- (ex-2 -| margin)
        node[right,scale=0.8,font=\color{sgreen}] (value) {true};
      \draw[pointy arrow, rounded corners] ([shift={(1mm,1mm)}] template-2) -| (value);
      \path[transform canvas={yshift=1mm}] (template-2) -- (template-2 -| value)
        node[midway, above, scale=0.8] {{\color{sblue}T} = {\color{sgreen}int}};
    }
    \only<4>{
      \draw[pointy arrow,transform canvas={yshift=1mm}] ([xshift=1mm] ex-3) -- (ex-3 -| margin)
        node[right,scale=0.8,font=\color{sgreen}] (value) {true};
      \draw[pointy arrow, rounded corners] ([shift={(1mm,1mm)}] template-2) -| (value);
      \path[transform canvas={yshift=1mm}] (template-2) -- (template-2 -| value)
        node[midway, above, scale=0.8] {{\color{sblue}T} = {\color{sgreen}int}*};
    }
    \only<5>{
      \draw[pointy arrow,transform canvas={yshift=1mm}] ([xshift=1mm] ex-4) -- (ex-4 -| margin)
        node[right,scale=0.8,font=\color{sorange}] (value) {false};
      \draw[pointy arrow, rounded corners] ([shift={(1mm,1mm)}] template-1) -| (value);
      \path[transform canvas={yshift=1mm}] (template-1) -- (template-1 -| value)
        node[midway, above, scale=0.8] {{\color{sblue}T} = %
            {\color{sblue}std}::{\color{sgreen}shared\_ptr}<{\color{sgreen}int}>};
    }
    \only<6>{
      \draw[pointy arrow,transform canvas={yshift=1mm}] ([xshift=1mm] ex-4) -- (ex-4 -| margin)
        node[right,scale=0.8,font=\color{sgreen}] (value) {true};
      \draw[pointy arrow, rounded corners] ([shift={(1mm,1mm)}] template-3) -| (value);
      \path[transform canvas={yshift=1mm}] (template-3) -- (template-3 -| value)
        node[midway, above, scale=0.8] {{\color{sblue}T} = {\color{sgreen}int}};
    }

  \end{tikzpicture}

\end{frame}

\begin{frame}[fragile]
  \frametitle{Template specialisation : partial}

    {\changefontsize{12pt}
    Template deduction picks out the most specialised template}

    \vspace*{-.25cm}
  
    \begin{overlayarea}{\textwidth}{5cm}
    \begin{lstlisting}[basicstyle=\codefontsize{10pt}]
template <typename T>
struct num_ptr
{ static constexpr std::size_t value = 0; };
  \end{lstlisting}

  \begin{onlyenv}<1-3>
    \begin{lstlisting}[basicstyle=\codefontsize{10pt}]
template <typename T>
struct num_ptr<T*>(*\tmark{template-1}*)
{ ... value = 1; };

template <typename T>
struct num_ptr<T**>(*\tmark{template-2}*)
{ ... value = 2; };
    \end{lstlisting}
  \end{onlyenv}

  \begin{onlyenv}<4->
    \begin{lstlisting}[basicstyle=\codefontsize{10pt}]
template <typename T>
struct num_ptr<T*>
{ ... value = 1 + num_ptr<T>::value; };
    \end{lstlisting}
  \end{onlyenv}

  \begin{onlyenv}<5->
    \begin{lstlisting}[basicstyle=\codefontsize{10pt}]
template <typename T>
struct num_ptr<std::shared_ptr<T>>
{ ... value = 1 + num_ptr<T>::value; };
    \end{lstlisting}
  \end{onlyenv}

  \end{overlayarea}

  \begin{overlayarea}{\textwidth}{2cm}
    \begin{onlyenv}<1-3>
      \begin{lstlisting}[basicstyle=\codefontsize{10pt}]
num_ptr<int*>::value;(*\tmark{ex-1}*)
num_ptr<int**>::value;(*\tmark{ex-2}*)
      \end{lstlisting}
    \end{onlyenv}

    \begin{onlyenv}<6->
      \begin{lstlisting}[basicstyle=\codefontsize{10pt}]
num_ptr<std::shared_ptr<int**>***>::value;(*\tmark{ex-3}*)
      \end{lstlisting}
    \end{onlyenv}
  \end{overlayarea}
    
  \nointerlineskip
  \begin{tikzpicture}[overlay,remember picture]
    \coordinate (margin) at ([xshift=-5cm] current page.south east);
    \only<2>{
      \draw[pointy arrow,transform canvas={yshift=1mm}] ([xshift=1mm] ex-1) -- (ex-1 -| margin)
        node[right,scale=0.8] (value) {{\color{sblue}value} = 1};
      \draw[pointy arrow, rounded corners] ([shift={(1mm,1mm)}] template-1) -| (value);
      \path[transform canvas={yshift=1mm}] (template-1) -- (template-1 -| value)
        node[midway, above, scale=0.8] {{\color{sblue}T} = {\color{sgreen}int}};
    }
    \only<3>{
      \draw[pointy arrow,transform canvas={yshift=1mm}] ([xshift=1mm] ex-2) -- (ex-2 -| margin)
        node[right,scale=0.8] (value) {{\color{sblue}value} = 2};
      \draw[pointy arrow, rounded corners] ([shift={(1mm,1mm)}] template-2) -| (value);
      \path[transform canvas={yshift=1mm}] (template-2) -- (template-2 -| value)
        node[midway, above, scale=0.8] {{\color{sblue}T} = {\color{sgreen}int}};
    }
    \only<7>{
      \draw[pointy arrow,transform canvas={yshift=1mm}] ([xshift=1mm] ex-3) -- +(.5cm,0)
        node[right,scale=0.8] (value) {{\color{sblue}value} = 6};

      \node[anchor=south] at ([yshift=.25cm] current page.south) %
        [font=\itshape\changefontsize{10pt}] {I am getting ahead of myself};
    }
  \end{tikzpicture}

\end{frame}

\begin{frame}[fragile]
  \frametitle{Template specialisation : partial}

  Only class templates have partial specialisation

  \vspace{1cm}

  Cannot have it for function templates due to overloading

\end{frame}

\begin{frame}

  \begin{center}\fontsize{16pt}{16pt}\selectfont\color{Tropiteal}
    Fundamental theorem of templates
  \end{center}

  \vspace{.25cm}
  \begin{center}
    \begin{minipage}{8cm}
      Any finite problem can be solved with a finite number of structs.
    \end{minipage}
  \end{center}

  \vspace{.25cm}
  \begin{flushright}\color{ICantExpress!50!WhiteTrash}
    Jonas R. Glesaaen
  \end{flushright}

\end{frame}

\begin{frame}[fragile]
  \frametitle{Template specialisation : partial}

  \begin{lstlisting}[basicstyle=\codefontsize{10pt}]
template <typename T>
struct function_impl
{ 
  static auto _(T t) { ... };
};

template <typename T>
struct function_impl<T[]>
{
  static auto _(T[] t) { ... };
};

template <typename T>
auto foo(T t)
{
  return function_impl<T>::_(t);
}
  \end{lstlisting}

\end{frame}

\section{Summary and wrap up}
\frame{\sectionpage}

\begin{frame}[fragile]
  \frametitle{Template feature table}

  \vspace*{-.5cm}

  \begin{tikzpicture}[remember picture,
    yes/.style={
      draw, scale=0.75, fill=AtomicBikini,
      text width=1.75cm,align=center
    },
    no/.style={
      draw, scale=0.75, fill=Marty,
      text width=1.75cm,align=center
    }]
    \matrix [
    nodes={font=\itshape\changefontsize{12pt}},
    column sep=.75cm,
    ] (complexity)
    {
      & \node[align=center] {Type\\deduction}; & \node[align=center] {Full\\spec.}; & \node[align=center] {Partial\\spec.}; \\
      \node {Function}; & \node[yes] {Yes}; & \node[yes] {Yes}; & \node[no] {No}; \\
      \node {Class}; & \node[yes] {C++17}; & \node[yes] {Yes}; & \node[yes] {Yes}; \\
      \node {Alias}; & \node[no] {No}; & \node[no] {No}; & \node[no] {No}; \\
      \node {Variable}; & \node[no] {No}; & \node[yes] {Yes}; & \node[yes] {Yes}; \\
    };
  \end{tikzpicture}

\end{frame}

\begin{frame}
  \frametitle{References}

  \frametitle{Resources}
  \bibliographystyle{plain}
  {\footnotesize
    \bibliography{ref}}

\end{frame}

\begin{frame}

  \begin{center}

    {\changefontsize{28pt}\color{Tropiteal}Thanks!}

    \vspace{1cm}

    \begin{tikzpicture}
      \node[scale=0.075] (cat) {\includegraphics{figures/Octocat.png}};
      \node[right of=cat, anchor=west, font=\changefontsize{18pt}] {Irubataru/talks};
    \end{tikzpicture}
  \end{center}
  
\end{frame}

\end{document}
