\documentclass[14pt,a4paper,dvipsnames,usenames]{beamer}
%\documentclass[14pt,a4paper,dvipsnames,usenames,handout]{beamer}
\usepackage[T1]{fontenc}
\usepackage{url}
\usepackage{xcolor}
\usepackage{amsmath,varwidth}
\usepackage{tikz}
\usepackage{listings,algpseudocode}
\usepackage[quiet]{mathspec}

\setmainfont[
  ItalicFont={Yanone Kaffeesatz Light},
  Scale=1.3,
  LetterSpace=2.0
]{Yanone Kaffeesatz Bold}

\setmonofont{Hack}

% The base16 colour scheme?
\definecolor{sbase03}{HTML}{002B36}
\definecolor{sbase02}{HTML}{073642}
\definecolor{sbase01}{HTML}{586E75}
\definecolor{sbase00}{HTML}{657B83}
\definecolor{sbase0}{HTML}{839496}
\definecolor{sbase1}{HTML}{93A1A1}
\definecolor{sbase2}{HTML}{EEE8D5}
\definecolor{sbase3}{HTML}{FDF6E3}
\definecolor{syellow}{HTML}{B58900}
\definecolor{sorange}{HTML}{CB4B16}
\definecolor{sred}{HTML}{DC322F}
\definecolor{smagenta}{HTML}{D33682}
\definecolor{sviolet}{HTML}{6C71C4}
\definecolor{sblue}{HTML}{268BD2}
\definecolor{scyan}{HTML}{2AA198}
\definecolor{sgreen}{HTML}{859900}

\definecolor{Tropiteal}{RGB}{0,168,198}
\definecolor{TealDrop}{RGB}{64,192,203}
\definecolor{WhiteTrash}{RGB}{249,242,231}
\definecolor{AtomicBikini}{RGB}{174,226,57}
\definecolor{FeebleWeek}{RGB}{143,190,0}
\definecolor{ICantExpress}{RGB}{28,20,13}
\definecolor{Marty}{RGB}{250,42,0}

\colorlet{ColourBase}{Tropiteal}
\colorlet{ColourHl1}{Marty}
\colorlet{ColourHl2}{FeebleWeek}
\colorlet{ColourHl3}{TealDrop}
\colorlet{ColourDark}{ICantExpress}
\colorlet{ColourDark2}{Tropiteal}


\usetheme{metropolis}

\setbeamercolor{alerted text}{%
  fg=bazelGreen
}

\setbeamercolor{example text}{%
  fg=mLightBrown
}

\setbeamercolor{frametitle}{%
  use=normal text,
  fg=normal text.bg,
  bg=bazelGreen
}

\lstset{%
  basicstyle=\ttfamily\lst@ifdisplaystyle\fontsize{9pt}{9pt}\selectfont\fi,
  keywordstyle=\color{sgreen}, identifierstyle=\color{sblue}, 
  commentstyle=\color{sbase1}, stringstyle=\color{sorange},
  numberstyle=\color{sviolet}, showstringspaces=false,  
  breaklines=true,
  tabsize=5,
}

\lstalias[]{gnumake}[gnu]{make}

% -------------------------- %
% Standard LaTeX definitions %
% -------------------------- %

\DeclareGraphicsExtensions{.pdf}
\renewcommand\textbullet{\ensuremath{\bullet}}

% ----------------------------------- %
% Various tikz styles and definitions %
% ----------------------------------- %

\usetikzlibrary{calc,intersections,positioning,matrix,chains,scopes}
\usetikzlibrary{decorations.pathmorphing,decorations.pathreplacing,shapes}
\newcommand{\tikzmark}[1]{\tikz[overlay,remember picture] \node (#1) {};}

\tikzumlset{fill class=RoyalBlue!20}

\tikzstyle{startnode} = [circle, draw, fill=Bittersweet!20, minimum size=10pt, inner sep = 0]
\tikzstyle{stopnode} = [startnode]
\tikzstyle{process} = [rectangle, minimum width=2cm, minimum height=.7cm, text centered, draw, fill=OliveGreen!20, align=center, font=\scriptsize]
\tikzstyle{dangerprocess} = [process, fill=BrickRed!20, draw=BrickRed, thick]
\tikzstyle{flowarrow} = [->, >=stealth, thick]

% ---------------------------------- %
% My default listings C++ code style %
% ---------------------------------- %

\lstset{%
  language=C++, basicstyle=\scriptsize\ttfamily, 
  keywordstyle=\color{OliveGreen}, identifierstyle=\color{RoyalBlue}, 
  commentstyle=\color{Brown}, stringstyle=\color{Bittersweet}, showstringspaces=false,  
  breaklines=true, prebreak=\mbox{{\color{Bittersweet}\tiny\ $\hookleftarrow$}}, 
  postbreak=\mbox{{\color{Bittersweet}\tiny$\to$\ }}, tabsize=5,
  morekeywords={%
    MyClass,unique_ptr,shared_ptr,weak_ptr,auto_ptr,
    make_unique, make_shared, default_delete,
    dynamic_pointer_cast, const_pointer_cast,
    scoped_ptr,QSharedPointer,QScopedPointer,QWeakPointer,
    ftring, SmartPtr, Shape, Circle, ShapeFactory, CircleFactory,
    Derived, Base
  }
}

\def\inline{\lstinline[basicstyle=\ttfamily\normalsize]}

% ---------------------------------------------------------------------------------------------------- %
% Various command for inline code instead of lstinline so that I don't have to update the keyword list %
% ---------------------------------------------------------------------------------------------------- %

\newcommand{\object}[1]{{\ttfamily \color{OliveGreen}#1}}
\newcommand{\function}[1]{{\ttfamily \color{RoyalBlue}#1}}
\newcommand{\std}[1]{{\ttfamily {\color{RoyalBlue} std::}{\color{OliveGreen}#1}}}
\newcommand{\boost}[1]{{\ttfamily {\color{RoyalBlue} boost::}{\color{OliveGreen}#1}}}

\newcommand{\uniqueptr}{\std{unique\_ptr}}
\newcommand{\sharedptr}{\std{shared\_ptr}}
\newcommand{\weakptr}{\std{weak\_ptr}}
\newcommand{\autoptr}{\std{auto\_ptr}}

\newcommand{\dimtext}[2]%
{
  { \transparent{0.7}
  \begin{tikzpicture}[overlay, remember picture]
    \fill[white] ( #1 -| current page.north west) -- ++(0,.8em) -- ++(\paperwidth,0) -- (#2 -| current page.north east)
   -- ++(0,-.5em) -- ++(-\paperwidth,0) -- cycle;
  \end{tikzpicture}
  }
}

\newcommand{\specialcell}[2][c]{%
  \begin{tabular}
    [#1]{@{}c@{}}#2
  \end{tabular}
}

\newcommand{\tikzmark}[1]{\tikz[overlay,remember picture] \coordinate (#1);}

\newcommand{\qmat}{\ensuremath{Q_{\mathrm{uark}}}}
\newcommand{\mathfont}{\fontsize{10pt}{12pt}}
\newcommand{\minus}{\scalebox{0.75}[1.0]{$-$}}
\newcommand{\sectionframe}{%
  \addtocounter{framenumber}{-1}%
  \frame[plain]{\begin{center}\LARGE \color{beameralert} \insertsection\end{center}}}

\newcommand{\breakline}{%
  \begin{center} \begin{tikzpicture}
      \draw[beamerprimary] ({-0.5 * \textwidth},0) -- ({0.5 * \textwidth},0);
      \node[inner sep=0, minimum size=7pt, fill=white,circle] {};
      \node[inner sep=0, minimum size=4pt, draw=beamerprimary,circle] {};
      \node[inner sep=0, minimum size=1pt, fill=beamerprimary,circle] {};
  \end{tikzpicture} \end{center}
}


\usefonttheme{serif}

%\setbeameroption{show notes}

% Turn off page numbering
\setbeamertemplate{footline}{}

\title{Thinking with templates}
\author{\texorpdfstring{%
    Jonas Rylund Glesaaen\newline\fontsize{12pt}{12pt}\selectfont\texttt{jonas@glesaaen.com}%
  }{%
    Jonas Rylund Glesaaen}}
\date{\texorpdfstring{%
    C++ User Group Meeting\newline{}January 27th 2016%
    }{%
      January 27th 2016}}

\begin{document}

\nocite{*}

\frame{\titlepage}

\begin{frame}[plain]
  \nointerlineskip
  \begin{tikzpicture}[overlay,remember picture]
    \fill[Marty] (current page.north east) rectangle (current page.south west);
    \node at ([yshift=-.25em] current page.center) [font=\LARGE, WhiteTrash] {Disclaimer};
  \end{tikzpicture}
\end{frame}

\begin{frame}
  \frametitle{Today's topics}

  \tableofcontents
  
\end{frame}

\begin{frame}

  \begin{center}
    {\fontsize{24pt}{24pt}\selectfont\color{Tropiteal}Goal}

    \vspace{1cm}
    Write code that is {\color{FeebleWeek}easy} to use {\color{FeebleWeek}correctly}\\
    but {\color{Marty}hard} to use {\color{Marty}incorrectly}
  \end{center}
  
\end{frame}

\section{Understanding types}

\frame[plain]{\sectionpage}

\begin{frame}[fragile]
  \frametitle{The information stored in types}

  
  \begin{onlyenv}<1|handout:1>
  \begin{center}
      \begin{lstlisting}[basicstyle=\ttfamily]
  double power(double, int);
      \end{lstlisting}
  \end{center}
  \end{onlyenv}

  \note<1|handout:1>{
    This function accepts a single double and a single integer. If something else is passed to the function it will be converted
    to doubles and integers first, the function knows nothing about this.
    
    \vspace{.5em}
    Since the function copies the arguments passed to it it will have to know the size of the objects, which restricts its
    behaviour.
  }

  \begin{onlyenv}<2|handout:2>
  \begin{center}
      \begin{lstlisting}[basicstyle=\ttfamily]
  void start(Widget &);
      \end{lstlisting}
  \end{center}
  \end{onlyenv}

  \note<2|handout:2>{
    This function accepts a reference to a Widget, meaning that it will accept anything that looks sufficiently like a Widget
    object (including children).
    
    \vspace{.5em}
    The only information the function uses about Widget is its explicit interface.
  }

  \begin{onlyenv}<3-4|handout:3>
    The type tells the compiler

    \vspace{1em}
    \begin{itemize}
    \setlength\itemsep{.5cm}
      \item How much space an object needs in memory
      \item {\only<4>{\color{Marty}}What operations can be carried out on the object}
    \end{itemize}

    \vspace{1cm}
    {\fontsize{10pt}{10pt}\selectfont{}(lets ignore type specifiers for now)}
    
  \end{onlyenv}

  \begin{onlyenv}<5-|handout:4>
  \begin{center}
      \begin{lstlisting}[basicstyle=\ttfamily]
  Widget w = Widget{} + Widget{};
      \end{lstlisting}
  \end{center}

  \only<6|handout:4>{%
    \vspace{1cm}
    What are the requirements on the \lstinline!Widget! type for this line to compile?%
  }
  \end{onlyenv}

  \note<6|handout:4>{%
    Whether this is a valid "expression" or not is information 100\% stored in the Widget type and is completely independent of
    the runtime dynamics of the code.
  }
  
\end{frame}

\begin{frame}[fragile]
  \frametitle{Declaring typenames}
  \framesubtitle{Class declaration}

  \begin{lstlisting}[basicstyle=\ttfamily]
struct my_struct { ... };
  \end{lstlisting}

  \vspace{.5cm}
  \begin{lstlisting}[basicstyle=\ttfamily]
class my_class { ... };
  \end{lstlisting}

  \vspace{.5cm}
  \begin{lstlisting}[basicstyle=\ttfamily]
enum class my_enum { ... };
  \end{lstlisting}

  \vspace{.75cm}
  Can be compared to variable declaration
  
\end{frame}

\begin{frame}[fragile]
  \frametitle{Declaring typenames}
  \framesubtitle{Class declaration}

  \begin{lstlisting}[basicstyle=\ttfamily]
class my_class
  : public my_base_class { ... };
  \end{lstlisting}

  \vspace{.3cm}
  Reads:\\[5pt]
  "The class {\color{Tropiteal}my\_class} is a {\color{Tropiteal}my\_base\_class}"

  \begin{onlyenv}<2|handout:1>
  \vspace{1cm}
  Much like variable assignment

  \vspace{.1cm}
  \begin{lstlisting}[basicstyle=\ttfamily,morekeywords={var}]
var my_var = my_base_var;
  \end{lstlisting}
  \end{onlyenv}

\end{frame}

\begin{frame}[fragile]
  \frametitle{Declaring typenames}
  \framesubtitle{Type aliasing}

  \begin{lstlisting}[basicstyle=\ttfamily]
typedef std::vector<double> Vec;
  \end{lstlisting}

  \vspace{.5cm}
  \begin{lstlisting}[basicstyle=\ttfamily]
using WidgetFunction =
  std::function<void(Widget&)>;
  \end{lstlisting}
  
\end{frame}

\begin{frame}[fragile]
  \frametitle{Explicit interfaces}

  \begin{lstlisting}[escapeinside={(*}{*)},deletekeywords={Widget}]
class Widget
{
public:
  using value_type = double;
  using refernce = double&;
    \end{lstlisting}
    \begin{lstlisting}[escapeinside={(*}{*)},morekeywords={reference, value_type,size_t}]
  Widget();
  ~Widget();

  reference operator[](std::size_t);
  value_type calculate() const;

  void swap(Widget &);
};
  \end{lstlisting}
  
\end{frame}

\begin{frame}[fragile]
  \frametitle{Nested types}

  Types can be nested inside of other compound types

  \vspace{.3cm}
  \begin{lstlisting}[deletekeywords={Widget}]
class Widget
{
  class iterator { ... };

  enum class AccessType { ... };

  using pointer = std::unique_ptr<DataType>;
  
  ...
};
  \end{lstlisting}

  \vspace{.3cm}
  They provide an extended explicit interface
  
\end{frame}

\begin{frame}[fragile]
  \frametitle{Nested types}
  \framesubtitle{Example: Type array}

  \begin{lstlisting}[morekeywords={WidgetFunction,string,nullptr_t}]
struct TypeArray
{
  using one = double;
  using two = std::string;
  using three = WidgetFunction;
  using four = std::list<int>;
  using five = std::nullptr_t;
  ...
};
  \end{lstlisting}

\end{frame}

\begin{frame}[fragile]
  \frametitle{Nested types}
  \framesubtitle{Example: Type map}

  \vspace*{-.3cm}
  \begin{lstlisting}[morekeywords={function,string}]
struct TypeMap
{
  struct one
  {
    using first = int;
    using second = std::string;
  };

  struct two
  {
    using first = double**;
    using second = function<void(double**)>;
  };
  ...
};
  \end{lstlisting}

\end{frame}

\begin{frame}[fragile]
  \frametitle{Nested types}

  The nested types can be accessed using {\color{Tropiteal}::}

  \vspace{1cm}
  \begin{lstlisting}[basicstyle=\ttfamily]
using x = TypeArray::three;
using y = TypeMap::two::first;
  \end{lstlisting}

\end{frame}

\begin{frame}[fragile]
  \frametitle{Nested types}

  Nested types should provide information about its parent type

  \vspace{.3cm}
  \begin{onlyenv}<1>
  \begin{lstlisting}[deletekeywords={Widget},morekeywords={DataType}]
class Widget
{
  using pointer = DataType*;
  \end{lstlisting}
  \end{onlyenv}
  \begin{onlyenv}<2>
  \begin{lstlisting}[deletekeywords={Widget},morekeywords={DataType}]
class Widget
{
  using pointer = std::shared_ptr<DataType>;
  \end{lstlisting}
  \end{onlyenv}
  \begin{lstlisting}[morekeywords={pointer}]
  pointer data() const;

  ...
};

Widget w {};
Widget::pointer data_ptr = w.data();
  \end{lstlisting}
  
\end{frame}

\begin{frame}[fragile]
  \frametitle{Why should you care?}

  Types is the language your compiler speaks

  \vspace{.5cm}
  \begin{itemize}
    \setlength\itemsep{.25cm}
    \item Better control over the compiler through type manipulation
    \item Type specific logic errors to be checked at compile time
    \item Can wirte code that are optimized and readable at the same time
  \end{itemize}
  
\end{frame}

\section{Transcending types}

\frame[plain]{\sectionpage}

\begin{frame}[fragile]
  \frametitle{Types as compile time variables}

  Standard scenario: dereference \& swap

  \vspace{.25cm}
  \begin{lstlisting}
using VecIterator
  = std::vector<double>::iterator;
  \end{lstlisting}
  \begin{lstlisting}[morekeywords={VecIterator}]
void
iterator_swap(VecIterator i1, VecIterator i2)
{
  double tmp = *i1;
  *i1 = *i2;
  *i2 = tmp;
}
  \end{lstlisting}

  \vspace{.25cm}
  It would be natural that this function should work for all iterators that can be dereferenced \& assigned
  
\end{frame}

\begin{frame}[fragile]
  \frametitle{Types as compile time variables}

  \begin{lstlisting}[escapeinside={(*}{*)}]
template <typename InputIt1, typename InputIt2>
  \end{lstlisting}
  \vspace*{-.4cm}
  \begin{lstlisting}[escapeinside={(*}{*)},morekeywords={InputIt1,InputIt2}]
void iterator_swp(InputIt1 it1, InputIt2 it2)
{
     (*\tikzmark{missingType}*)    tmp = *it1;

  *it1 = *it2;
  *it2 = tmp;
}
  \end{lstlisting}

  \vspace{.5cm}
  Have we lost information?

  \vspace{.5cm}
  The compiler should know what type {\color{Tropiteal}*it1} gives when instansiating the template function 

  \nointerlineskip
  \begin{tikzpicture}[overlay,remember picture]
    \node at ([yshift=.1cm]missingType) [scale=1.5,color=Marty]{?};
  \end{tikzpicture}
  
\end{frame}

\begin{frame}[fragile]
  \frametitle{Types as compile time variables}

  This information that can be stored in nested types

  \vspace{.5cm}
  \begin{lstlisting}[escapeinside={(*}{*)}]
template <typename InputIt1, typename InputIt2>
  \end{lstlisting}
  \vspace*{-.4cm}
  \begin{lstlisting}[morekeywords={InputIt1,InputIt2,value_type}]
void iterator_swp(InputIt1 it1, InputIt2 it2)
{
  using deref_type
    = typename InputIt1::value_type;
  \end{lstlisting}
  \begin{lstlisting}[morekeywords={deref_type}]
  deref_type tmp = *it1;
  *it1 = *it2;
  *it2 = tmp;
}
  \end{lstlisting}

  \vspace{.5cm}
  But no way to make it compatible with built in types
\end{frame}

\begin{frame}

  \begin{center}\fontsize{16pt}{16pt}\selectfont\color{Tropiteal}
    Fundamental theorem of software engineering
  \end{center}

  \vspace{.25cm}
  \begin{center}
    \begin{minipage}{8cm}
      We can solve any problem by introducing an extra level of indirection.
    \end{minipage}
  \end{center}

  \vspace{.25cm}
  \begin{flushright}\color{ICantExpress!50!WhiteTrash}
    David J. Wheeler
  \end{flushright}

\end{frame}

\begin{frame}[fragile]
  \frametitle{Types as compile time variables}

  \begin{lstlisting}
template <typename Iterator>
struct iterator_traits
{
  using value_type
  \end{lstlisting}
  \vspace*{-.4cm}
  \begin{lstlisting}[morekeywords={Iterator,value_type}]
    = typename Iterator::value_type;
  ...
};
  \end{lstlisting}
  \begin{lstlisting}
template <typename Ptr>
struct iterator_traits<Ptr*>
{
  \end{lstlisting}
  \vspace*{-.4cm}
  \begin{lstlisting}[morekeywords={Ptr}]
  using value_type = Ptr;
  ...
};
  \end{lstlisting}

  \uncover<2|handout:1>{
  \vspace{.25cm}
  blah blah blah...
  }

  \note<2|handout:1>{
    There should exist a better solution, the information is already there
  }

\end{frame}

\begin{frame}
  \frametitle{Automatic type deduction}

  {\large\color{FeebleWeek}auto}
  
  \vspace{.1cm}
  Used as a type definition, {\fontsize{10pt}{10pt}\selectfont{}(mostly)} carries out standard template type deduction on the
  right hand side of the assignment operator

  \vspace{1cm}
  {\large\color{FeebleWeek}decltype}

  \vspace{.1cm}
  Given a name or an expression, returns the name's or expression's type
  
\end{frame}

\begin{frame}[fragile]
  \frametitle{Automatic type deduction}

  We can use {\color{FeebleWeek}auto} to fix our type issue

  \vspace{.5cm}
  \begin{lstlisting}
template <typename InputIt1, typename InputIt2>
  \end{lstlisting}
  \vspace*{-.4cm}
  \begin{lstlisting}[morekeywords={InputIt1,InputIt2,value_type}]
void iterator_swp(InputIt1 it1, InputIt2 it2)
{
  auto tmp = *it1;
  *it1 = *it2;
  *it2 = tmp;
}
  \end{lstlisting}

\end{frame}

\begin{frame}[fragile]
  \frametitle{Return type deduction}

  \begin{lstlisting}
template <typename Func, typename Type>
  \end{lstlisting}
  \vspace*{-.2cm}
  \begin{lstlisting}[escapeinside={(*}{*)},morekeywords={Func,Type},deletekeywords={map}]
(*\tikzmark{unknownReturn}*)        map_function(Func f, Type val)
{

  return f.apply(val)(*\tikzmark{returnValue}*);
}
  \end{lstlisting}

  \nointerlineskip
  \begin{tikzpicture}[overlay,remember picture]
    \node at ([shift={(.8cm,.1cm)}]unknownReturn) [scale=1.5,color=Marty]{?};
    \only<2>{
    \draw[pointy arrow] ([shift={(-5ex,-.75ex)}] returnValue) .. controls +(0,-.5cm) and +(-.5cm,0) .. +(0,-1cm)
      node[right,scale=0.75] {%
        We want whatever type this expression returns
      };
    }
  \end{tikzpicture}
   
\end{frame}

\begin{frame}[fragile]
  \frametitle{Return type deduction}
  \framesubtitle{Attempt one}

  We know basically what we want

  \vspace{.5cm}
  \begin{lstlisting}
template <typename Func, typename Type>
  \end{lstlisting}
  \vspace*{-.4cm}
  \begin{lstlisting}[escapeinside={(*}{*)},morekeywords={Func,Type,decltype},deletekeywords={map}]
decltype(f.apply(val)) (*\tikzmark{decltype}*)
map_function(Func f, Type val)
{
  return f.apply(val);
}
  \end{lstlisting}

  \vspace{.5cm}
  But how do we wrangle this information from the compiler?

  \only<2>{
  \nointerlineskip
  \begin{tikzpicture}[overlay,remember picture]
    \draw[pointy arrow] ([yshift=.35ex] decltype) -- +(3.5cm,0)
      node[right,align=left,scale=0.75] {%
        names {\color{Tropiteal}f} and {\color{Tropiteal}val}\\
        not yet declared
      };
  \end{tikzpicture}
  }
   
\end{frame}

\begin{frame}[fragile]
  \frametitle{Return type deduction}

  \begin{onlyenv}<1|handout:1>
  C++11 trailing return type

  \vspace{.5cm}
  \begin{lstlisting}
template <typename Func, typename Type>
  \end{lstlisting}
  \vspace*{-.4cm}
  \begin{lstlisting}[morekeywords={Func,Type,decltype},deletekeywords={map}]
auto map_function(Func f, Type val)
  -> decltype(f.apply(val))
{
  return f.apply(val);
}
  \end{lstlisting}
  \end{onlyenv}

  \begin{onlyenv}<2-3|handout:2>
  C++14 automatic return type deduction

  \vspace{.5cm}
  \begin{lstlisting}
template <typename Func, typename Type>
  \end{lstlisting}
  \vspace*{-.4cm}
  \begin{lstlisting}[morekeywords={Func,Type,decltype},deletekeywords={map}]
auto map_function(Func f, Type val)
{
  return f.apply(val);
}
  \end{lstlisting}

  \vspace{.5cm}
  \uncover<3|handout:2>{
    Might not always produce the expected return type
  }
  \end{onlyenv}

\end{frame}

\begin{frame}[fragile]
  \frametitle{Template type deduction}

  Assume the following piece of template pseudocode

  \vspace{1cm}
  \lstinline!template <typename! {\ttfamily\bfseries\color{Marty}Type}\lstinline!>!\\
  \lstinline!void foo(! {\ttfamily\bfseries\color{Marty}ParamType}\lstinline! param);!\\[10pt]
  \lstinline!foo(! {\ttfamily\bfseries\itshape\color{Marty}expr}\lstinline!);!

  \vspace{1cm}
  The deduced types of {\ttfamily\bfseries\color{Marty}Type} and {\ttfamily\bfseries\color{Marty}ParamType} from
  {\ttfamily\bfseries\itshape\color{Marty}expr} depends on the form of {\itshape{}ParamType}
  
\end{frame}

\begin{frame}[fragile]
  \frametitle{Template type deduction}

  \lstinline!template <typename! {\ttfamily\bfseries\color{Marty}Type}\lstinline!>!\\
  \lstinline!void foo(! {\ttfamily\bfseries\color{Marty}ParamType}\lstinline! param);!\\[5pt]
  \lstinline!foo(! {\ttfamily\bfseries\itshape\color{Marty}expr}\lstinline!);!

  \vspace{.5cm}
  \begin{overlayarea}{\textwidth}{6cm}
  \only<1|handout:1>{
  Case 1:\\[5pt]
  {ParamType} is a reference or a pointer\\{\fontsize{10pt}{10pt}\selectfont{}(but not a \&\& reference)}

  \vspace{.5cm}
  \begin{enumerate}
    \item Ignore reference part of expr
    \item Pattern-match expr's type with ParamType to deduce Type
  \end{enumerate}
  }

  \only<2|handout:2>{
  Case 2:\\[5pt]
  {ParamType} is a universal reference

  \vspace{.5cm}
  \begin{itemize}
    \item If expr is an lvalue reference, ParamType will be deduced to be an lvalue reference
    \item If expr is an rvalue reference, standard rules apply
  \end{itemize}
  }

  \only<3|handout:3>{
  Case 3:\\[5pt]
  {ParamType} is not a pointer nor a reference

  \vspace{.5cm}
  \begin{enumerate}
    \item Ignore reference and const part of expr
    \item Pattern-match expr's type with ParamType to deduce Type
  \end{enumerate}
  }
  \end{overlayarea}

\end{frame}

\begin{frame}[fragile]
  \frametitle{Return type deduction}

  \begin{onlyenv}<1>
  This will slice any references from the return type

  \vspace{.5cm}
  \begin{lstlisting}
template <typename Func, typename Type>
  \end{lstlisting}
  \vspace*{-.4cm}
  \begin{lstlisting}[morekeywords={Func,Type,decltype},deletekeywords={map}]
auto map_function(Func f, Type val)
{
  return f.apply(val);
}
  \end{lstlisting}
  \end{onlyenv}

  \begin{onlyenv}<2>
  Using {\color{FeebleWeek}decltype} will pattern match correctly

  \vspace{.5cm}
  \begin{lstlisting}
template <typename Func, typename Type>
  \end{lstlisting}
  \vspace*{-.4cm}
  \begin{lstlisting}[morekeywords={Func,Type,decltype},deletekeywords={map}]
decltype(auto) map_function(Func f, Type val)
{
  return f.apply(val);
}
  \end{lstlisting}
  \end{onlyenv}

\end{frame}

\begin{frame}[fragile]
  \frametitle{Template pattern matching}

  Can use the pattern matching to extract types from templates

  \vspace{.5cm}
  \begin{lstlisting}[basicstyle=\ttfamily]
template <typename Type>
  \end{lstlisting}
  \vspace*{-.4cm}
  \begin{lstlisting}[basicstyle=\ttfamily,morekeywords={Type}]
Type extract(Widget<Type>) { ... }
  \end{lstlisting}
  
\end{frame}

\begin{frame}[fragile]
  \frametitle{Template pattern matching}

  ...or the other way around

  \vspace{.5cm}
  \begin{lstlisting}
template <
  template Other,
  template <typename...> class Policy
>
  \end{lstlisting}
  \vspace*{-.4cm}
  \begin{lstlisting}[morekeywords={Policy, Other}]
Policy<Other> replace(Policy<Widget>) { ... }
  \end{lstlisting}
  
\end{frame}

\begin{frame}[fragile]
  \frametitle{Template pattern matching}

  Also good for restricting pattern matching when you know what patterns you expect to be valid

  \vspace{.5cm}
  \begin{lstlisting}
template <
  template <typename...> class CreationPolicy
>
  \end{lstlisting}
  \vspace*{-.4cm}
  \begin{lstlisting}[morekeywords={CreationPolicy}]
class WidgetManager 
  : public CreationPolicy<Widget> 
{ ... }
  \end{lstlisting}
  
\end{frame}

\begin{frame}[fragile]
  \frametitle{Template pattern matching}

  Note that no implicit conversions are considered during type deduction

  \vspace{.5cm}
  \begin{lstlisting}[escapeinside={(*}{*)}]
template <typename T>
void fill(std::vector<T> &v, T x); 

std::vector<double> vec(6);
fill(vec, 1); (*\tikzmark{compileError}*)
  \end{lstlisting}

  \vspace{.5cm}
  \begin{overlayarea}{\textwidth}{2cm}
  \only<1>{\ttfamily\fontsize{8pt}{8pt}\selectfont%
    {\color{Marty}error}: no matching function for call to 'fill'\\
    \hspace{.3cm}fill(vec, 1);\\
    \hspace{.3cm}{\color{FeebleWeek}\^{}---}\\
    note: candidate template ignored: deduced conflicting types for parameter 'T' ('double' vs. 'int')
  }
  \only<2>{
  Not even between built in types
  }
  \end{overlayarea}

  \note<2>{
    This is good as it means you can be more restrictive with your pattern matching, making sure to write code that only compiles
    with the types you want
  }

\end{frame}

\begin{frame}[fragile]
  \frametitle{Template pattern matching}

  Recursive pattern matching for variadic templates

  \vspace{.5cm}

  \begin{lstlisting}[basicstyle=\ttfamily\fontsize{8pt}{8pt}\selectfont,morekeywords={ostream}]
void println(std::ostream & os)
{
  os << std::endl;
}

template <typename H, typename... T>
  \end{lstlisting}
  \vspace*{-.4cm}
  \begin{lstlisting}[basicstyle=\ttfamily\fontsize{8pt}{8pt}\selectfont,morekeywords={H,T,ostream}]
void println(std::ostream & os, const H & head, T... tail)
{
  os << head;
  if( sizeof...(tail) != 0)
    os << ", ";

  println(os,tail...);
}


println(std::cout, 7, 8.43, 'c', "Hello");
  \end{lstlisting}
  
\end{frame}

\begin{frame}[fragile]
  \frametitle{Encoding intent in types}

  The C++ Guidelines have suggested a new template type to signal resource ownership

  \vspace{.5cm}
  \begin{lstlisting}[basicstyle=\ttfamily]
template <typename T>
using owner = T;
  \end{lstlisting}

  \vspace{.5cm}
  \begin{enumerate}
    \item The code signals the intent of the programmer
    \item Can be checked by the compiler
  \end{enumerate}
  
\end{frame}

\begin{frame}[fragile]
  \frametitle{Encoding intent in types}

  \begin{lstlisting}[escapeinside={(*}{*)},morekeywords={owner}]
owner<Widget*> FactoryMethod() {...}; (*\tikzmark{factory}*)

Widget* w1 = FactoryMethod(); (*\tikzmark{err1}*)

Widget* w2 = new Widget {}; (*\tikzmark{err2}*)


auto w3 = FactoryMethod();
Widget* w4 = w3; (*\tikzmark{ok1}*)
...
delete w4; (*\tikzmark{err3}*)
  \end{lstlisting}

  \nointerlineskip
  \begin{tikzpicture}[overlay,remember picture]
    \only<1>{
    \draw[pointy arrow] ([yshift=.35ex] factory) -- +(.5cm,0)
      node[right,align=left,scale=0.65] {%
        factory methods need\\
        to return owner types
      };
    }
    \only<2>{
    \draw[pointy arrow] ([yshift=.35ex] err1) -- +(1cm,0)
      node[right,align=left,scale=0.65] {%
        {\color{Marty}error:} information on\\
        ownership lost
      };
    }
    \only<3>{
    \draw[pointy arrow] ([yshift=.35ex] err2) -- +(1cm,0)
      node[right,align=left,scale=0.65] {%
        {\color{Marty}error:} cannot assign newed\\
        objects to non-owners
      };
    }
    \only<4>{
    \draw[pointy arrow] ([yshift=.35ex] ok1) -- +(3cm,0)
      node[right,align=left,scale=0.65] {%
        {\color{FeebleWeek}ok:} a raw pointer simply\\
        points to something
      };
    }
    \only<5>{
    \draw[pointy arrow] ([yshift=.35ex] err3) -- +(3cm,0)
      node[right,align=left,scale=0.65] {%
        {\color{Marty}error:} cannot\\
        delete non-owners
      };
    }
  \end{tikzpicture}
  
\end{frame}

\begin{frame}[fragile]
  \frametitle{Implicit interfaces}

  \begin{lstlisting}[deletekeywords={Widget}]
template <typename Widget, typename Operator>
  \end{lstlisting}
  \vspace*{-.4cm}
  \begin{lstlisting}[morekeywords={Operator}]
void check_and_apply(Widget &w, Operator op)
{
  if (w.size() > 10 and !w.bad())
    op.apply(w);
}
  \end{lstlisting}

  \vspace{.5cm}
  The expressions in the functon body make up the template's implicit interface

  
\end{frame}

\begin{frame}[fragile]
  \frametitle{Curiously recurring template pattern}

  \begin{lstlisting}[basicstyle=\ttfamily]
template <typename T>
class Base { ... };

class Derived
  \end{lstlisting}
  \vspace*{-.4cm}
  \begin{lstlisting}[basicstyle=\ttfamily,morekeywords={Base,Derived}]
  : public Base<Derived> { ... };
  \end{lstlisting}
  
\end{frame}

\begin{frame}[fragile]
  \frametitle{Curiously recurring template pattern}

  Allows for static polymorphism

  %\vspace{.2cm}
  \begin{overlayarea}{\textwidth}{6.5cm}
  \begin{onlyenv}<1>
  \begin{lstlisting}
template <typename Type>
class Base
{
public:
  Type& self()
  {
    return static_cast<Type&>(*this);
  }

  void implementation()
  {
    self().implementation();
  }
};

  \end{lstlisting}
  \end{onlyenv}

  \begin{onlyenv}<2>
  \begin{lstlisting}
class Derived
  \end{lstlisting}
  \vspace*{-.4cm}
  \begin{lstlisting}[morekeywords={Base}]
  : public Base<Derived>
{
public:
  void implementation() { ... };
};

template <typename Type>
void call(Base<Type> widget)
{
  widget.implementation();
}
  \end{lstlisting}
  \end{onlyenv}
  \end{overlayarea}
  
\end{frame}

\begin{frame}[fragile]
  \frametitle{Curiously recurring template pattern}

  Makes it easier to put things in a common box
  \vspace{.2cm}
  
  \begin{overlayarea}{\textwidth}{6.5cm}
  \begin{onlyenv}<1>
  \begin{lstlisting}[basicstyle=\ttfamily\fontsize{8pt}{8pt}\selectfont]
template <typename Val>
class unary { ... };

template <typename LVal, typename RVal>
class binary { ... };
  \end{lstlisting}
  %\vspace*{-.4cm}
  \begin{lstlisting}[basicstyle=\ttfamily\fontsize{8pt}{8pt}\selectfont,morekeywords={unary,binary}]
template <typename LValU, RValU>
auto operate(unary<LValU> left, unary<RValU> right)
 -> binary<unary<LValU>, unary<RValU>> { ... };

template <typename LValU, RValBL, RValBR>
auto operate(unary<LValU> left, binary<RValBL,RValBR> right)
 -> binary<unary<LValU>, binary<RValBL,RValBR>> { ... };

...
  \end{lstlisting}
  \end{onlyenv}

  \begin{onlyenv}<2>
  \begin{lstlisting}[basicstyle=\ttfamily\fontsize{8pt}{8pt}\selectfont]
template <typename Type>
class base { ... };
  \end{lstlisting}
  \begin{lstlisting}[basicstyle=\ttfamily\fontsize{8pt}{8pt}\selectfont,morekeywords={base}]
template <typename Val>
class unary : public base<unary<Val>> { ... };

template <typename LVal, typename RVal>
class binary : public base<binary<LVal,RVal>> { ... };
  \end{lstlisting}
  %\vspace*{-.4cm}
  \begin{lstlisting}[basicstyle=\ttfamily\fontsize{8pt}{8pt}\selectfont,morekeywords={base,unary,binary}]
template <typename LExpr, RExpr>
auto operate(base<LExpr> left, base<RExpr> right)
 -> binary<LExpr,RExpr> { ... };
  \end{lstlisting}
  \end{onlyenv}
  \end{overlayarea}



\end{frame}

\begin{frame}[fragile]
  \frametitle{Building trees}

  \begin{lstlisting}[basicstyle=\ttfamily\fontsize{8pt}{8pt}\selectfont]
struct plus {};
struct minus {};
struct times {};
struct divide {};


template <typename Expr>
struct base_expr {};


template <typename Op, typename Le, typename Re>
  \end{lstlisting}
  \vspace*{-.4cm}
  \begin{lstlisting}[basicstyle=\ttfamily\fontsize{8pt}{8pt}\selectfont,morekeywords={base_expr,Op,Le,Re}]
struct binary_expr : base_expr<binary_expr<Op,Le,Re>> {};


struct val : base_expr<val> {};
  \end{lstlisting}
  
\end{frame}

\begin{frame}[fragile]
  \frametitle{Building trees}

  \begin{lstlisting}[basicstyle=\ttfamily\fontsize{8pt}{8pt}\selectfont,morekeywords={base_expr,binary_expr,val,plus}]
template <typename Le, typename Re>
auto operator+(base_expr<Le>, base_expr<Re>)
  -> binary_expr<plus,Le,Re>
{
  return {};
}

template <typename Le, typename Re>
auto operator-(base_expr<Le>, base_expr<Re>) {...}

template <typename Le, typename Re>
auto operator*(base_expr<Le>, base_expr<Re>) {...}

template <typename Le, typename Re> {...}
auto operator/(base_expr<Le>, base_expr<Re>) {...}

int main()
{
  val v;
  auto expr = v + v - v * v / (v + v);
}
  \end{lstlisting}
  
\end{frame}

\begin{frame}
  \frametitle{Resources}
  \bibliographystyle{plain}
  {\footnotesize
    \bibliography{ref}}
\end{frame}
  
\end{document}
